\documentclass{beamer}
\usepackage[italian]{babel}

\usepackage[utf8x]{inputenc}
\usepackage[T1]{fontenc}
\title{I numeri primi sono infiniti}
\author[Euclide]{Euclide di Alessandria \\
    \texttt{euclide@alessandria.edu}}
\date[VII SINP]{VII Simposio Internazionale sui Numeri Primi}
\institute[UniAlessandria]{Università di Alessandria}
\logo{
    \includegraphics[width=15mm]{sigillo}}
\usetheme{CambridgeUS}
% \useoutertheme[right]{sidebar}
% \usetheme{Madrid}
\usecolortheme{beaver}


% default
% albatross
% beaver
% beetle
% crane
% dolphin
% dove
% fly
% lily
% orchid
% rose
% seagull
% seahorse
% whale
% wolverine
% \useoutertheme{infolines}

% \lstset{basicstyle=\small}

\setbeamercovered{dynamic}
\theoremstyle{definition}
\newtheorem{definizione}{Definizione}

\theoremstyle{plain}
\newtheorem{teorema}{Teorema}
\begin{document}
\begin{frame}
    \maketitle
\end{frame}
\begin{frame}
    \frametitle{Piano della presentazione}
    \tableofcontents
\end{frame}
\section{Introduzione}
\begin{frame}
    \frametitle{Che cosa sono i numeri primi?}
    \begin{definizione}Un 
    \alert{numero primo} è un intero $>1$ che ha esattamente due divisori positivi.
\end{definizione}
\end{frame}
\section{L'infinità dei primi}
\begin{frame}
    \frametitle{I numeri primi sono infiniti}
    \framesubtitle{Ne diamo una dimostrazione diretta}
    \begin{teorema}Non esiste un primo maggiore di tutti gli altri.
\end{teorema}
    \pause
    \begin{proof}
    \begin{enumerate}[<+->]
    \item Sia dato un elenco di primi.
    \item Sia $q$ il loro prodotto.
    \item Allora $q+1$ è divisibile per un primo $p$ che non compare nell'elenco. 
    \qedhere
\end{enumerate}
\end{proof}
\end{frame}
\section{Problemi aperti}
\begin{frame}
    \frametitle{Che cosa c'è ancora da fare?}
    \begin{block}{Problemi risolti}Quanti sono i numeri primi?
\end{block}
    \begin{block}{Problemi aperti}Un numero pari $>2$ è sempre la somma di due primi?
\end{block}
\end{frame}
\end{document}