\chapter{Risultati}

\section{Metodo di valutazione}

Per ogni rete è stato osservato il numero di nodi infettati da ogni nodo facendo una media di \num{100} simulazioni e 
ottenendo così un vettore $s$ di $|V|$ elementi, dove $s_i$ rappresenta il numero di nodi 
infettati in media in uno \spproc\ in cui i nodi inizialmente infetti sono le controparti del 
nodo~$i$ nei layer in cui esso è presente.

È stato poi calcolato per ogni algoritmo un vettore $a$ dove $a_i$ 
è il punteggio assegnato dall'algoritmo al nodo~$i$.

Per confrontare i vettori così ottenuti ed assegnare un punteggio ad ogni algoritmo che ne 
valutasse la capacità di riconoscere gli \infsp\ è stato utilizzato coefficiente \textit{Kendall's}~$\tau$~\cite{kendall:tau}, 
così definito:

\begin{equation}
    \tau = \frac{n_c - n_d}{n(n-1)}
\end{equation}

dove:
\begin{itemize}
    \item $n$ è il numero di elementi dei due vettori, ossia il numero di nodi |V|;
    \item $n_c$ è il numero di coppie concordanti nei due vettori. Una coppia di elementi 
            indici $i, j$ si dice concordante se $a_i > a_j$ e $s_i > s_j$ oppure se $a_i < a_j$ e $s_i < s_j$;
    \item $n_d$ è il numero di coppie discordanti nei due vettori. Una coppia di elementi 
    indici $i, j$ si dice discordante se $a_i > a_j$ e $s_i < s_j$ oppure se $a_i < a_j$ e $s_i > s_j$;    
\end{itemize}

Se $a_i = a_j$ oppure $s_i = s_j$, la coppia $i, j$ non viene contata.

Il risultato della formula è compreso tra \num{-1.0} e {1.0}

\section{Confronto dei risultati}

Le tabelle \ref{tab:taudln}, \ref{tab:tausln} e \ref{tab:taumux} mostrano i punteggi 
ottenuti da ogni algoritmo nelle reti DLN, SLN e Multiplex rispettivamente.
In ogni tabella sono evidenziati per ogni rete i punteggi ottenuti dai \num{3} algoritmi 
migliori. 

Si può notare che \textit{mlPCI} e \textit{laPCI} sono i due algoritmi che ottengono punteggi buoni 
su tutte le reti utilizzate, e possono quindi essere considerati un buon metodo per indentificare 
gli \infsp.

\begin{table}[!htbp]
    \centering
    \caption{Kendall's $\tau$ di diversi algoritmi in reti DLN}
    \label{tab:taudln}
    \begin{tabular}{lrrrrrrrr}
        \toprule
          & DLN$_{\begin{subarray}{l} 0.3\text{,} \\ 0.3\text{,} \\ 0.3 \end{subarray}}$ 
          & DLN$_{\begin{subarray}{l} 0.3\text{,} \\ 0.3\text{,} \\ 0.8 \end{subarray}}$ 
          & DLN$_{\begin{subarray}{l} 0.3\text{,} \\ 0.8\text{,} \\ 0.3 \end{subarray}}$ 
          & DLN$_{\begin{subarray}{l} 0.3\text{,} \\ 0.8\text{,} \\ 0.8 \end{subarray}}$ 
          & DLN$_{\begin{subarray}{l} 0.8\text{,} \\ 0.3\text{,} \\ 0.3 \end{subarray}}$ 
          & DLN$_{\begin{subarray}{l} 0.8\text{,} \\ 0.3\text{,} \\ 0.8 \end{subarray}}$ 
          & DLN$_{\begin{subarray}{l} 0.8\text{,} \\ 0.8\text{,} \\ 0.3 \end{subarray}}$ 
          & DLN$_{\begin{subarray}{l} 0.8\text{,} \\ 0.8\text{,} \\ 0.8 \end{subarray}}$ \\
            
        \midrule
    %         addPR  &   {\num{ 0.4885}} &   {\num{ 0.4844}} &   {\num{ 0.4888}} &   {\num{ 0.4860}} &   {\num{ 0.4872}} &   {\num{ 0.4874}} &   {\num{ 0.4853}} &   {\num{ 0.4848}} \\
    %     addPR$^T$  &   {\num{ 0.3649}} &   {\num{ 0.4137}} &   {\num{ 0.3601}} &   {\num{ 0.4134}} &   {\num{ 0.3744}} &   {\num{ 0.4136}} &   {\num{ 0.3755}} &   {\num{ 0.4110}} \\
    %       aggCore  &   {\num{ 0.5575}} &   {\num{ 0.5376}} &   {\num{ 0.5552}} &   {\num{ 0.5379}} &   {\num{ 0.5596}} &   {\num{ 0.5391}} &   {\num{ 0.5578}} &   {\num{ 0.5416}} \\
    %   aggCore$^T$  &   {\num{ 0.6063}} &   {\num{ 0.5604}} &   {\num{ 0.5946}} &   {\num{ 0.5740}} &   {\num{ 0.5718}} &   {\num{ 0.5698}} &   {\num{ 0.5823}} &   {\num{ 0.5809}} \\
    %        aggDeg  &   {\num{ 0.6620}} &   {\num{ 0.6402}} &   {\num{ 0.6607}} &   {\num{ 0.6401}} &   {\num{ 0.6274}} &   {\num{ 0.6201}} &   {\num{ 0.6297}} &   {\num{ 0.6179}} \\
    %         aggPR  &   {\num{ 0.5316}} &   {\num{ 0.4958}} &   {\num{ 0.5296}} &   {\num{ 0.4972}} &   {\num{ 0.5214}} &   {\num{ 0.4879}} &   {\num{ 0.5195}} &   {\num{ 0.4869}} \\
    %     aggPR$^T$  &   {\num{ 0.6161}} &   {\num{ 0.5469}} &   {\num{ 0.6115}} &   {\num{ 0.5461}} &   {\num{ 0.5484}} &   {\num{ 0.4965}} &   {\num{ 0.5496}} &   {\num{ 0.4930}} \\
    %         alPCI  &   {\num{ 0.6945}} & \3{\num{ 0.7036}} & \3{\num{ 0.7043}} & \3{\num{ 0.7109}} &   {\num{ 0.6565}} &   {\num{ 0.6694}} &   {\num{ 0.6776}} &   {\num{ 0.6800}} \\
    %         laPCI  & \2{\num{ 0.7145}} & \2{\num{ 0.7157}} & \2{\num{ 0.7113}} & \2{\num{ 0.7158}} & \3{\num{ 0.6759}} & \2{\num{ 0.6910}} & \3{\num{ 0.6783}} & \3{\num{ 0.6868}} \\
    %         lsPCI  &   {\num{ 0.6313}} &   {\num{ 0.6285}} &   {\num{ 0.6354}} &   {\num{ 0.6335}} & \2{\num{ 0.6840}} & \3{\num{ 0.6848}} & \2{\num{ 0.6879}} & \2{\num{ 0.6895}} \\
    %         mlPCI  & \1{\num{ 0.7344}} & \1{\num{ 0.7406}} & \1{\num{ 0.7305}} & \1{\num{ 0.7427}} & \1{\num{ 0.6980}} & \1{\num{ 0.7114}} & \1{\num{ 0.7027}} & \1{\num{ 0.7095}} \\
    %     multiCore  &   {\num{ 0.3551}} &   {\num{ 0.3574}} &   {\num{ 0.3554}} &   {\num{ 0.3576}} &   {\num{ 0.3551}} &   {\num{ 0.3549}} &   {\num{ 0.3553}} &   {\num{ 0.3550}} \\
    % multiCore$^T$  &   {\num{ 0.3646}} &   {\num{ 0.3635}} &   {\num{ 0.3647}} &   {\num{ 0.3632}} &   {\num{ 0.3648}} &   {\num{ 0.3633}} &   {\num{ 0.3649}} &   {\num{ 0.3644}} \\
    %       sumCore  &   {\num{ 0.5225}} &   {\num{ 0.5350}} &   {\num{ 0.5259}} &   {\num{ 0.5346}} &   {\num{ 0.5279}} &   {\num{ 0.5350}} &   {\num{ 0.5259}} &   {\num{ 0.5334}} \\
    %   sumCore$^T$  &   {\num{ 0.4377}} &   {\num{ 0.4813}} &   {\num{ 0.4340}} &   {\num{ 0.4824}} &   {\num{ 0.4468}} &   {\num{ 0.4816}} &   {\num{ 0.4474}} &   {\num{ 0.4804}} \\
    %         verBC  &   {\num{ 0.6586}} &   {\num{ 0.5605}} &   {\num{ 0.6610}} &   {\num{ 0.5639}} &   {\num{ 0.6275}} &   {\num{ 0.5404}} &   {\num{ 0.6309}} &   {\num{ 0.5441}} \\
    %         verPR  &   {\num{ 0.5275}} &   {\num{ 0.5141}} &   {\num{ 0.5261}} &   {\num{ 0.5136}} &   {\num{ 0.5195}} &   {\num{ 0.5078}} &   {\num{ 0.5162}} &   {\num{ 0.5064}} \\
    %     verPR$^T$  & \3{\num{ 0.6985}} &   {\num{ 0.6547}} &   {\num{ 0.6934}} &   {\num{ 0.6535}} &   {\num{ 0.6645}} &   {\num{ 0.6554}} &   {\num{ 0.6647}} &   {\num{ 0.6364}} \\
               addPR &   {\num{ 0.4885}} &   {\num{ 0.4844}} &   {\num{ 0.4888}} &   {\num{ 0.4860}} &   {\num{ 0.4872}} &   {\num{ 0.4874}} &   {\num{ 0.4853}} &   {\num{ 0.4848}} \\
             aggCore &   {\num{ 0.5575}} &   {\num{ 0.5376}} &   {\num{ 0.5552}} &   {\num{ 0.5379}} &   {\num{ 0.5596}} &   {\num{ 0.5391}} &   {\num{ 0.5578}} &   {\num{ 0.5416}} \\
              aggDeg &   {\num{ 0.6620}} &   {\num{ 0.6402}} &   {\num{ 0.6607}} &   {\num{ 0.6401}} &   {\num{ 0.6274}} &   {\num{ 0.6201}} &   {\num{ 0.6297}} &   {\num{ 0.6179}} \\
               aggPR &   {\num{ 0.5316}} &   {\num{ 0.4958}} &   {\num{ 0.5296}} &   {\num{ 0.4972}} &   {\num{ 0.5214}} &   {\num{ 0.4879}} &   {\num{ 0.5195}} &   {\num{ 0.4869}} \\
               alPCI & \3{\num{ 0.6945}} & \3{\num{ 0.7036}} & \3{\num{ 0.7043}} & \3{\num{ 0.7109}} &   {\num{ 0.6565}} &   {\num{ 0.6694}} &   {\num{ 0.6776}} &   {\num{ 0.6800}} \\
               laPCI & \2{\num{ 0.7145}} & \2{\num{ 0.7157}} & \2{\num{ 0.7113}} & \2{\num{ 0.7158}} & \3{\num{ 0.6759}} & \2{\num{ 0.6910}} & \3{\num{ 0.6783}} & \3{\num{ 0.6868}} \\
               lsPCI &   {\num{ 0.6313}} &   {\num{ 0.6285}} &   {\num{ 0.6354}} &   {\num{ 0.6335}} & \2{\num{ 0.6840}} & \3{\num{ 0.6848}} & \2{\num{ 0.6879}} & \2{\num{ 0.6895}} \\
               mlPCI & \1{\num{ 0.7344}} & \1{\num{ 0.7406}} & \1{\num{ 0.7305}} & \1{\num{ 0.7427}} & \1{\num{ 0.6980}} & \1{\num{ 0.7114}} & \1{\num{ 0.7027}} & \1{\num{ 0.7095}} \\
           multiCore &   {\num{ 0.3551}} &   {\num{ 0.3574}} &   {\num{ 0.3554}} &   {\num{ 0.3576}} &   {\num{ 0.3551}} &   {\num{ 0.3549}} &   {\num{ 0.3553}} &   {\num{ 0.3550}} \\
             sumCore &   {\num{ 0.5225}} &   {\num{ 0.5350}} &   {\num{ 0.5259}} &   {\num{ 0.5346}} &   {\num{ 0.5279}} &   {\num{ 0.5350}} &   {\num{ 0.5259}} &   {\num{ 0.5334}} \\
               verBC &   {\num{ 0.6586}} &   {\num{ 0.5605}} &   {\num{ 0.6610}} &   {\num{ 0.5639}} &   {\num{ 0.6275}} &   {\num{ 0.5404}} &   {\num{ 0.6309}} &   {\num{ 0.5441}} \\
               verPR &   {\num{ 0.5275}} &   {\num{ 0.5141}} &   {\num{ 0.5261}} &   {\num{ 0.5136}} &   {\num{ 0.5195}} &   {\num{ 0.5078}} &   {\num{ 0.5162}} &   {\num{ 0.5064}} \\
        \bottomrule
    \end{tabular}
    % \\[10 pt] %You can adjust how far below the table the text should appear
    % Coefficienti Kendall's $\tau$ ottenuti dai diversi algoritmi sulle reti DLN generate.
    % Per ogni rete sono evidenziati i punteggi ottenuti dai 3 algoritmi migliori.
    % \begin{tabular} 
    %     Should be a caption \\
    % \end{tabular}
\end{table}

% # dln
%          DLN_2_0.3_0.3_0.3
% ['mlPCI', 'laPCI', 'verPR_T']
%          DLN_2_0.3_0.3_0.8
% ['mlPCI', 'laPCI', 'alPCI']
%          DLN_2_0.3_0.8_0.3
% ['mlPCI', 'laPCI', 'alPCI']
%          DLN_2_0.3_0.8_0.8
% ['mlPCI', 'laPCI', 'alPCI']
%          DLN_2_0.8_0.3_0.3
% ['mlPCI', 'lsPCI', 'laPCI']
%          DLN_2_0.8_0.3_0.8
% ['mlPCI', 'laPCI', 'lsPCI']
%          DLN_2_0.8_0.8_0.3
% ['mlPCI', 'lsPCI', 'laPCI']
%          DLN_2_0.8_0.8_0.8
% ['mlPCI', 'lsPCI', 'laPCI']
%       addPR 0.4885 0.4844 0.4888 0.4860 0.4872 0.4874 0.4853 0.4848 
%     addPR_T 0.3649 0.4137 0.3601 0.4134 0.3744 0.4136 0.3755 0.4110 
%     aggCore 0.5575 0.5376 0.5552 0.5379 0.5596 0.5391 0.5578 0.5416 
%   aggCore_T 0.6063 0.5604 0.5946 0.5740 0.5718 0.5698 0.5823 0.5809 
%      aggDeg 0.6620 0.6402 0.6607 0.6401 0.6274 0.6201 0.6297 0.6179 
%       aggPR 0.5316 0.4958 0.5296 0.4972 0.5214 0.4879 0.5195 0.4869 
%     aggPR_T 0.6161 0.5469 0.6115 0.5461 0.5484 0.4965 0.5496 0.4930 
%       alPCI 0.6945 0.7036 0.7043 0.7109 0.6565 0.6694 0.6776 0.6800 
%       laPCI 0.7145 0.7157 0.7113 0.7158 0.6759 0.6910 0.6783 0.6868 
%       lsPCI 0.6313 0.6285 0.6354 0.6335 0.6840 0.6848 0.6879 0.6895 
%       mlPCI 0.7344 0.7406 0.7305 0.7427 0.6980 0.7114 0.7027 0.7095 
%   multiCore 0.3551 0.3574 0.3554 0.3576 0.3551 0.3549 0.3553 0.3550 
% multiCore_T 0.3646 0.3635 0.3647 0.3632 0.3648 0.3633 0.3649 0.3644 
%     sumCore 0.5225 0.5350 0.5259 0.5346 0.5279 0.5350 0.5259 0.5334 
%   sumCore_T 0.4377 0.4813 0.4340 0.4824 0.4468 0.4816 0.4474 0.4804 
%       verBC 0.6586 0.5605 0.6610 0.5639 0.6275 0.5404 0.6309 0.5441 
%       verPR 0.5275 0.5141 0.5261 0.5136 0.5195 0.5078 0.5162 0.5064 
%     verPR_T 0.6985 0.6547 0.6934 0.6535 0.6645 0.6554 0.6647 0.6364



% # dln
%          DLN_2_0.3_0.3_0.3
% ['mlPCI', 'laPCI', 'alPCI']
%          DLN_2_0.3_0.3_0.8
% ['mlPCI', 'laPCI', 'alPCI']
%          DLN_2_0.3_0.8_0.3
% ['mlPCI', 'laPCI', 'alPCI']
%          DLN_2_0.3_0.8_0.8
% ['mlPCI', 'laPCI', 'alPCI']
%          DLN_2_0.8_0.3_0.3
% ['mlPCI', 'lsPCI', 'laPCI']
%          DLN_2_0.8_0.3_0.8
% ['mlPCI', 'laPCI', 'lsPCI']
%          DLN_2_0.8_0.8_0.3
% ['mlPCI', 'lsPCI', 'laPCI']
%          DLN_2_0.8_0.8_0.8
% ['mlPCI', 'lsPCI', 'laPCI']
%       addPR 0.4885 0.4844 0.4888 0.4860 0.4872 0.4874 0.4853 0.4848 
%     aggCore 0.5575 0.5376 0.5552 0.5379 0.5596 0.5391 0.5578 0.5416 
%      aggDeg 0.6620 0.6402 0.6607 0.6401 0.6274 0.6201 0.6297 0.6179 
%       aggPR 0.5316 0.4958 0.5296 0.4972 0.5214 0.4879 0.5195 0.4869 
%       alPCI 0.6945 0.7036 0.7043 0.7109 0.6565 0.6694 0.6776 0.6800 
%       laPCI 0.7145 0.7157 0.7113 0.7158 0.6759 0.6910 0.6783 0.6868 
%       lsPCI 0.6313 0.6285 0.6354 0.6335 0.6840 0.6848 0.6879 0.6895 
%       mlPCI 0.7344 0.7406 0.7305 0.7427 0.6980 0.7114 0.7027 0.7095 
%   multiCore 0.3551 0.3574 0.3554 0.3576 0.3551 0.3549 0.3553 0.3550 
%     sumCore 0.5225 0.5350 0.5259 0.5346 0.5279 0.5350 0.5259 0.5334 
%       verBC 0.6586 0.5605 0.6610 0.5639 0.6275 0.5404 0.6309 0.5441 
%       verPR 0.5275 0.5141 0.5261 0.5136 0.5195 0.5078 0.5162 0.5064
\begin{table}[!htbp]
    \caption{Kendall's $\tau$ di diversi algoritmi in reti SLN}
    \label{tab:tausln}
    \centering
    \begin{tabular}{lrrrrrrrr}
        \toprule
          & SLN$_{\begin{subarray}{l} 0.3\text{,} \\ 0.3\text{,} \\ 0.3 \end{subarray}}$ 
          & SLN$_{\begin{subarray}{l} 0.3\text{,} \\ 0.3\text{,} \\ 0.8 \end{subarray}}$ 
          & SLN$_{\begin{subarray}{l} 0.3\text{,} \\ 0.8\text{,} \\ 0.3 \end{subarray}}$ 
          & SLN$_{\begin{subarray}{l} 0.3\text{,} \\ 0.8\text{,} \\ 0.8 \end{subarray}}$ 
          & SLN$_{\begin{subarray}{l} 0.8\text{,} \\ 0.3\text{,} \\ 0.3 \end{subarray}}$ 
          & SLN$_{\begin{subarray}{l} 0.8\text{,} \\ 0.3\text{,} \\ 0.8 \end{subarray}}$ 
          & SLN$_{\begin{subarray}{l} 0.8\text{,} \\ 0.8\text{,} \\ 0.3 \end{subarray}}$ 
          & SLN$_{\begin{subarray}{l} 0.8\text{,} \\ 0.8\text{,} \\ 0.8 \end{subarray}}$ \\
        \midrule
    %         addPR  &     {\num{ 0.4225}} &     {\num{ 0.4293}} &     {\num{ 0.4216}} &     {\num{ 0.4295}} &     {\num{ 0.4198}} &     {\num{ 0.4309}} &     {\num{ 0.4178}} &     {\num{ 0.4287}} \\
    %     addPR$^T$  &     {\num{ 0.4674}} &     {\num{ 0.4831}} &     {\num{ 0.4714}} &     {\num{ 0.4829}} &     {\num{ 0.4675}} &     {\num{ 0.4824}} &     {\num{ 0.4676}} &     {\num{ 0.4849}} \\
    %       aggCore  &     {\num{ 0.4066}} &     {\num{ 0.3684}} &     {\num{ 0.4158}} &     {\num{ 0.3731}} &     {\num{ 0.4010}} &     {\num{ 0.3795}} &     {\num{ 0.4117}} &     {\num{ 0.3841}} \\
    %   aggCore$^T$  &     {\num{ 0.4608}} &     {\num{ 0.4225}} &     {\num{ 0.4601}} &     {\num{ 0.4300}} &     {\num{ 0.5110}} &     {\num{ 0.4362}} &     {\num{ 0.4811}} &     {\num{ 0.4663}} \\
    %        aggDeg  &     {\num{ 0.6073}} &     {\num{ 0.5520}} &     {\num{ 0.6009}} &     {\num{ 0.5460}} &     {\num{ 0.6147}} &     {\num{ 0.5490}} &     {\num{ 0.6134}} &     {\num{ 0.5412}} \\
    %         aggPR  &     {\num{ 0.4275}} &     {\num{ 0.3619}} &     {\num{ 0.4313}} &     {\num{ 0.3695}} &     {\num{ 0.4229}} &     {\num{ 0.3713}} &     {\num{ 0.4294}} &     {\num{ 0.3779}} \\
    %     aggPR$^T$  &     {\num{ 0.5900}} &     {\num{ 0.5700}} &     {\num{ 0.5822}} &     {\num{ 0.5625}} &     {\num{ 0.5814}} &     {\num{ 0.5413}} &     {\num{ 0.5815}} &     {\num{ 0.5362}} \\
    %         alPCI  & \Fst{\num{ 0.6550}} & \Trd{\num{ 0.6152}} & \Snd{\num{ 0.6379}} & \Trd{\num{ 0.5971}} & \Snd{\num{ 0.6675}} & \Snd{\num{ 0.6300}} & \Trd{\num{ 0.6555}} & \Trd{\num{ 0.6203}} \\
    %         laPCI  &     {\num{ 0.6243}} &     {\num{ 0.5761}} &     {\num{ 0.6195}} &     {\num{ 0.5687}} & \Fst{\num{ 0.6685}} &     {\num{ 0.6208}} & \Fst{\num{ 0.6662}} &     {\num{ 0.6092}} \\
    %         lsPCI  &     {\num{ 0.5901}} &     {\num{ 0.5681}} &     {\num{ 0.5728}} &     {\num{ 0.5479}} &     {\num{ 0.5962}} &     {\num{ 0.5678}} &     {\num{ 0.5897}} &     {\num{ 0.5584}} \\
    %         mlPCI  & \Trd{\num{ 0.6265}} &     {\num{ 0.5810}} &     {\num{ 0.6222}} &     {\num{ 0.5771}} & \Trd{\num{ 0.6660}} &     {\num{ 0.6169}} & \Snd{\num{ 0.6627}} &     {\num{ 0.6057}} \\
    %     multiCore  &     {\num{ 0.3481}} &     {\num{ 0.3560}} &     {\num{ 0.3498}} &     {\num{ 0.3576}} &     {\num{ 0.3506}} &     {\num{ 0.3535}} &     {\num{ 0.3472}} &     {\num{ 0.3513}} \\
    % multiCore$^T$  &     {\num{ 0.4138}} &     {\num{ 0.3830}} &     {\num{ 0.4135}} &     {\num{ 0.4085}} &     {\num{ 0.4137}} &     {\num{ 0.4084}} &     {\num{ 0.4139}} &     {\num{ 0.4068}} \\
    %       sumCore  &     {\num{ 0.4209}} &     {\num{ 0.4288}} &     {\num{ 0.4211}} &     {\num{ 0.4273}} &     {\num{ 0.4164}} &     {\num{ 0.4291}} &     {\num{ 0.4171}} &     {\num{ 0.4267}} \\
    %   sumCore$^T$  & \Snd{\num{ 0.6339}} & \Fst{\num{ 0.6559}} & \Fst{\num{ 0.6382}} & \Fst{\num{ 0.6565}} &     {\num{ 0.6296}} & \Fst{\num{ 0.6545}} &     {\num{ 0.6316}} & \Fst{\num{ 0.6566}} \\
    %         verBC  &     {\num{ 0.5589}} &     {\num{ 0.4528}} &     {\num{ 0.5480}} &     {\num{ 0.4506}} &     {\num{ 0.5868}} &     {\num{ 0.4842}} &     {\num{ 0.5739}} &     {\num{ 0.4796}} \\
    %         verPR  &     {\num{ 0.3650}} &     {\num{ 0.3479}} &     {\num{ 0.3697}} &     {\num{ 0.3579}} &     {\num{ 0.3200}} &     {\num{ 0.3297}} &     {\num{ 0.3317}} &     {\num{ 0.3374}} \\
    %     verPR$^T$  &     {\num{ 0.6210}} & \Snd{\num{ 0.6245}} & \Trd{\num{ 0.6252}} & \Snd{\num{ 0.6251}} &     {\num{ 0.6323}} & \Trd{\num{ 0.6240}} &     {\num{ 0.6386}} & \Snd{\num{ 0.6207}} \\
              addPR  &     {\num{ 0.4225}} &     {\num{ 0.4293}} &     {\num{ 0.4216}} &     {\num{ 0.4295}} &     {\num{ 0.4198}} &     {\num{ 0.4309}} &     {\num{ 0.4178}} &     {\num{ 0.4287}} \\
            aggCore  &     {\num{ 0.4066}} &     {\num{ 0.3684}} &     {\num{ 0.4158}} &     {\num{ 0.3731}} &     {\num{ 0.4010}} &     {\num{ 0.3795}} &     {\num{ 0.4117}} &     {\num{ 0.3841}} \\
             aggDeg  &     {\num{ 0.6073}} &     {\num{ 0.5520}} &     {\num{ 0.6009}} &     {\num{ 0.5460}} &     {\num{ 0.6147}} &     {\num{ 0.5490}} &     {\num{ 0.6134}} &     {\num{ 0.5412}} \\
              aggPR  &     {\num{ 0.4275}} &     {\num{ 0.3619}} &     {\num{ 0.4313}} &     {\num{ 0.3695}} &     {\num{ 0.4229}} &     {\num{ 0.3713}} &     {\num{ 0.4294}} &     {\num{ 0.3779}} \\
              alPCI  & \Fst{\num{ 0.6550}} & \Fst{\num{ 0.6152}} & \Fst{\num{ 0.6379}} & \Fst{\num{ 0.5971}} & \Snd{\num{ 0.6675}} & \Fst{\num{ 0.6300}} & \Trd{\num{ 0.6555}} & \Fst{\num{ 0.6203}} \\
              laPCI  & \Trd{\num{ 0.6243}} & \Trd{\num{ 0.5761}} & \Trd{\num{ 0.6195}} & \Trd{\num{ 0.5687}} & \Fst{\num{ 0.6685}} & \Snd{\num{ 0.6208}} & \Fst{\num{ 0.6662}} & \Snd{\num{ 0.6092}} \\
              lsPCI  &     {\num{ 0.5901}} &     {\num{ 0.5681}} &     {\num{ 0.5728}} &     {\num{ 0.5479}} &     {\num{ 0.5962}} &     {\num{ 0.5678}} &     {\num{ 0.5897}} &     {\num{ 0.5584}} \\
              mlPCI  & \Snd{\num{ 0.6265}} & \Snd{\num{ 0.5810}} & \Snd{\num{ 0.6222}} & \Snd{\num{ 0.5771}} & \Trd{\num{ 0.6660}} & \Trd{\num{ 0.6169}} & \Snd{\num{ 0.6627}} & \Trd{\num{ 0.6057}} \\
          multiCore  &     {\num{ 0.3481}} &     {\num{ 0.3560}} &     {\num{ 0.3498}} &     {\num{ 0.3576}} &     {\num{ 0.3506}} &     {\num{ 0.3535}} &     {\num{ 0.3472}} &     {\num{ 0.3513}} \\
            sumCore  &     {\num{ 0.4209}} &     {\num{ 0.4288}} &     {\num{ 0.4211}} &     {\num{ 0.4273}} &     {\num{ 0.4164}} &     {\num{ 0.4291}} &     {\num{ 0.4171}} &     {\num{ 0.4267}} \\
              verBC  &     {\num{ 0.5589}} &     {\num{ 0.4528}} &     {\num{ 0.5480}} &     {\num{ 0.4506}} &     {\num{ 0.5868}} &     {\num{ 0.4842}} &     {\num{ 0.5739}} &     {\num{ 0.4796}} \\
              verPR  &     {\num{ 0.3650}} &     {\num{ 0.3479}} &     {\num{ 0.3697}} &     {\num{ 0.3579}} &     {\num{ 0.3200}} &     {\num{ 0.3297}} &     {\num{ 0.3317}} &     {\num{ 0.3374}} \\
        \bottomrule

    \end{tabular}
\end{table}

% # sln
%          SLN_2_0.3_0.3_0.3
% ['alPCI', 'sumCore_T', 'mlPCI']
%          SLN_2_0.3_0.3_0.8
% ['sumCore_T', 'verPR_T', 'alPCI']
%          SLN_2_0.3_0.8_0.3
% ['sumCore_T', 'alPCI', 'verPR_T']
%          SLN_2_0.3_0.8_0.8
% ['sumCore_T', 'verPR_T', 'alPCI']
%          SLN_2_0.8_0.3_0.3
% ['laPCI', 'alPCI', 'mlPCI']
%          SLN_2_0.8_0.3_0.8
% ['sumCore_T', 'alPCI', 'verPR_T']
%          SLN_2_0.8_0.8_0.3
% ['laPCI', 'mlPCI', 'alPCI']
%          SLN_2_0.8_0.8_0.8
% ['sumCore_T', 'verPR_T', 'alPCI']
%       addPR 0.4225 0.4293 0.4216 0.4295 0.4198 0.4309 0.4178 0.4287 
%     addPR_T 0.4674 0.4831 0.4714 0.4829 0.4675 0.4824 0.4676 0.4849 
%     aggCore 0.4066 0.3684 0.4158 0.3731 0.4010 0.3795 0.4117 0.3841 
%   aggCore_T 0.4608 0.4225 0.4601 0.4300 0.5110 0.4362 0.4811 0.4663 
%      aggDeg 0.6073 0.5520 0.6009 0.5460 0.6147 0.5490 0.6134 0.5412 
%       aggPR 0.4275 0.3619 0.4313 0.3695 0.4229 0.3713 0.4294 0.3779 
%     aggPR_T 0.5900 0.5700 0.5822 0.5625 0.5814 0.5413 0.5815 0.5362 
%       alPCI 0.6550 0.6152 0.6379 0.5971 0.6675 0.6300 0.6555 0.6203 
%       laPCI 0.6243 0.5761 0.6195 0.5687 0.6685 0.6208 0.6662 0.6092 
%       lsPCI 0.5901 0.5681 0.5728 0.5479 0.5962 0.5678 0.5897 0.5584 
%       mlPCI 0.6265 0.5810 0.6222 0.5771 0.6660 0.6169 0.6627 0.6057 
%   multiCore 0.3481 0.3560 0.3498 0.3576 0.3506 0.3535 0.3472 0.3513 
% multiCore_T 0.4138 0.3830 0.4135 0.4085 0.4137 0.4084 0.4139 0.4068 
%     sumCore 0.4209 0.4288 0.4211 0.4273 0.4164 0.4291 0.4171 0.4267 
%   sumCore_T 0.6339 0.6559 0.6382 0.6565 0.6296 0.6545 0.6316 0.6566 
%       verBC 0.5589 0.4528 0.5480 0.4506 0.5868 0.4842 0.5739 0.4796 
%       verPR 0.3650 0.3479 0.3697 0.3579 0.3200 0.3297 0.3317 0.3374 
%     verPR_T 0.6210 0.6245 0.6252 0.6251 0.6323 0.6240 0.6386 0.6207 

% # sln
%          SLN_2_0.3_0.3_0.3
% ['alPCI', 'mlPCI', 'laPCI']
%          SLN_2_0.3_0.3_0.8
% ['alPCI', 'mlPCI', 'laPCI']
%          SLN_2_0.3_0.8_0.3
% ['alPCI', 'mlPCI', 'laPCI']
%          SLN_2_0.3_0.8_0.8
% ['alPCI', 'mlPCI', 'laPCI']
%          SLN_2_0.8_0.3_0.3
% ['laPCI', 'alPCI', 'mlPCI']
%          SLN_2_0.8_0.3_0.8
% ['alPCI', 'laPCI', 'mlPCI']
%          SLN_2_0.8_0.8_0.3
% ['laPCI', 'mlPCI', 'alPCI']
%          SLN_2_0.8_0.8_0.8
% ['alPCI', 'laPCI', 'mlPCI']
%       addPR 0.4225 0.4293 0.4216 0.4295 0.4198 0.4309 0.4178 0.4287 
%     aggCore 0.4066 0.3684 0.4158 0.3731 0.4010 0.3795 0.4117 0.3841 
%      aggDeg 0.6073 0.5520 0.6009 0.5460 0.6147 0.5490 0.6134 0.5412 
%       aggPR 0.4275 0.3619 0.4313 0.3695 0.4229 0.3713 0.4294 0.3779 
%       alPCI 0.6550 0.6152 0.6379 0.5971 0.6675 0.6300 0.6555 0.6203 
%       laPCI 0.6243 0.5761 0.6195 0.5687 0.6685 0.6208 0.6662 0.6092 
%       lsPCI 0.5901 0.5681 0.5728 0.5479 0.5962 0.5678 0.5897 0.5584 
%       mlPCI 0.6265 0.5810 0.6222 0.5771 0.6660 0.6169 0.6627 0.6057 
%   multiCore 0.3481 0.3560 0.3498 0.3576 0.3506 0.3535 0.3472 0.3513 
%     sumCore 0.4209 0.4288 0.4211 0.4273 0.4164 0.4291 0.4171 0.4267 
%       verBC 0.5589 0.4528 0.5480 0.4506 0.5868 0.4842 0.5739 0.4796 
%       verPR 0.3650 0.3479 0.3697 0.3579 0.3200 0.3297 0.3317 0.3374 

\begin{table}[!htbp]
    \caption{Kendall's $\tau$ di diversi algoritmi in reti multiplex}
    \label{tab:taumux}
    \centering
    \begin{tabular}{lrrrrrr}
        \toprule
          & Drosophila & Homo & MA2013 & NYCM2014 
          & SacchCere & SacchPomb \\
        \midrule 
              addPR &     {\num{ 0.0398}} &     {\num{ 0.3424}} &     {\num{ 0.0059}} &     {\num{-0.1313}} &     {\num{ 0.3300}} &     {\num{ 0.3410}} \\
            aggCore &     {\num{ 0.0646}} &     {\num{ 0.4132}} &     {\num{ 0.0572}} &     {\num{-0.0453}} &     {\num{ 0.4384}} &     {\num{ 0.1046}} \\
             aggDeg & \Fst{\num{ 0.7355}} & \Snd{\num{ 0.7096}} & \Fst{\num{ 0.5711}} & \Snd{\num{ 0.6150}} & \Trd{\num{ 0.6886}} & \Snd{\num{ 0.7656}} \\
              aggPR &     {\num{ 0.0417}} &     {\num{ 0.3857}} &     {\num{ 0.0164}} &     {\num{-0.0771}} &     {\num{ 0.3944}} &     {\num{ 0.2562}} \\
              alPCI &     {\num{ 0.3682}} &     {\num{ 0.1588}} &     {\num{ 0.0493}} &     {\num{ 0.1124}} &     {\num{ 0.0455}} &     {\num{ 0.4502}} \\
              laPCI & \Trd{\num{ 0.6040}} & \Trd{\num{ 0.6859}} & \Snd{\num{ 0.5534}} & \Fst{\num{ 0.6178}} & \Snd{\num{ 0.6980}} & \Trd{\num{ 0.6853}} \\
              lsPCI &     {\num{ 0.0000}} &     {\num{ 0.0000}} &     {\num{ 0.0000}} &     {\num{ 0.0000}} &     {\num{ 0.0000}} &     {\num{ 0.0000}} \\
              mlPCI & \Snd{\num{ 0.6947}} & \Fst{\num{ 0.7191}} & \Trd{\num{ 0.5532}} & \Trd{\num{ 0.6028}} & \Fst{\num{ 0.7073}} & \Fst{\num{ 0.7729}} \\
          multiCore &     {\num{ 0.0497}} &     {\num{ 0.1962}} &     {\num{ 0.0428}} &     {\num{ 0.0230}} &     {\num{ 0.1925}} &     {\num{-0.0170}} \\
            sumCore &     {\num{ 0.0658}} &     {\num{ 0.4575}} &     {\num{ 0.0616}} &     {\num{-0.0487}} &     {\num{ 0.4546}} &     {\num{ 0.1000}} \\
              verBC &     {\num{ 0.5243}} &     {\num{ 0.5653}} &     {\num{ 0.1726}} &     {\num{ 0.2045}} &     {\num{ 0.5666}} &     {\num{ 0.6550}} \\
              verPR &     {\num{-0.3505}} &     {\num{-0.0606}} &     {\num{-0.2707}} &     {\num{-0.4107}} &     {\num{ 0.0004}} &     {\num{-0.4201}} \\
        \bottomrule
    \end{tabular}
\end{table}


\paragraph{Algoritmi applicati al grafo trasposto}
Le diverse versioni degli algoritmi di \textit{PageRank} e \textit{k-core} sono state applicate anche al grafo trasposto.

Nel caso di \textit{PageRank}, se nella definizione sul grafo originale un nodo è tanto più importante quanto 
più importanti sono i nodi che hanno un arco verso di esso, applicando l'algoritmo sul grafo trasposto si ottiene
una definizione per cui un nodo è tanto più importante quanto più importanti sono i nodi verso cui ha un arco.

Per quanto riguarda \textit{k-core}, invece, nella definizione sul grafo originale un \textit{k-core} di un grafo è un sottografo 
massimale in cui ogni nodo ha almeno \textit{k} archi entranti, applicando l'algoritmo sul grafo trasposto si ottiene 
una definizione analoga, in cui invece degli archi entranti si considerano gli archi uscenti.

Per entrambi gli algoritmi si è pensato che le definizioni di centralità ottenute applicando l'algoritmo 
sul grafo trasposto potessero individuare meglio gli \infsp. Le prestazioni di questi algoritmi, indicati 
con l'indice $^T$, sono mostrate nelle tabelle \ref{tab:taudlnT}, \ref{tab:tauslnT} e \ref{tab:taumuxT}.

In generale, si può osservare come nella maggior parte dei casi questi ottengano risultati migliori se confrontati con 
la rispettiva versione applicata al grafo originale. 

\begin{table}[!htbp]
    \centering
    \caption{Kendall's $\tau$ di algoritmi applicati al grafo trasposto in reti DLN}
    \label{tab:taudlnT}
    \begin{tabular}{lrrrrrrrr}
        \toprule
          & DLN$_{\begin{subarray}{l} 0.3\text{,} \\ 0.3\text{,} \\ 0.3 \end{subarray}}$ 
          & DLN$_{\begin{subarray}{l} 0.3\text{,} \\ 0.3\text{,} \\ 0.8 \end{subarray}}$ 
          & DLN$_{\begin{subarray}{l} 0.3\text{,} \\ 0.8\text{,} \\ 0.3 \end{subarray}}$ 
          & DLN$_{\begin{subarray}{l} 0.3\text{,} \\ 0.8\text{,} \\ 0.8 \end{subarray}}$ 
          & DLN$_{\begin{subarray}{l} 0.8\text{,} \\ 0.3\text{,} \\ 0.3 \end{subarray}}$ 
          & DLN$_{\begin{subarray}{l} 0.8\text{,} \\ 0.3\text{,} \\ 0.8 \end{subarray}}$ 
          & DLN$_{\begin{subarray}{l} 0.8\text{,} \\ 0.8\text{,} \\ 0.3 \end{subarray}}$ 
          & DLN$_{\begin{subarray}{l} 0.8\text{,} \\ 0.8\text{,} \\ 0.8 \end{subarray}}$ \\
            
        \midrule
          addPR$^T$ &    {\num{ 0.3649}} &   {\num{ 0.4137}} &   {\num{ 0.3601}} &   {\num{ 0.4134}} &   {\num{ 0.3744}} &   {\num{ 0.4136}} &   {\num{ 0.3755}} &   {\num{ 0.4110}} \\
        aggCore$^T$ &  \3{\num{ 0.6063}} & \2{\num{ 0.5604}} & \3{\num{ 0.5946}} & \2{\num{ 0.5740}} & \2{\num{ 0.5718}} & \2{\num{ 0.5698}} & \2{\num{ 0.5823}} & \2{\num{ 0.5809}} \\
          aggPR$^T$ &  \2{\num{ 0.6161}} & \3{\num{ 0.5469}} & \2{\num{ 0.6115}} & \3{\num{ 0.5461}} & \3{\num{ 0.5484}} & \3{\num{ 0.4965}} & \3{\num{ 0.5496}} & \3{\num{ 0.4930}} \\
      multiCore$^T$ &    {\num{ 0.3646}} &   {\num{ 0.3635}} &   {\num{ 0.3647}} &   {\num{ 0.3632}} &   {\num{ 0.3648}} &   {\num{ 0.3633}} &   {\num{ 0.3649}} &   {\num{ 0.3644}} \\
        sumCore$^T$ &    {\num{ 0.4377}} &   {\num{ 0.4813}} &   {\num{ 0.4340}} &   {\num{ 0.4824}} &   {\num{ 0.4468}} &   {\num{ 0.4816}} &   {\num{ 0.4474}} &   {\num{ 0.4804}} \\
          verPR$^T$ &  \1{\num{ 0.6985}} & \1{\num{ 0.6547}} & \1{\num{ 0.6934}} & \1{\num{ 0.6535}} & \1{\num{ 0.6645}} & \1{\num{ 0.6554}} & \1{\num{ 0.6647}} & \1{\num{ 0.6364}} \\
        \bottomrule
    \end{tabular}
    % \\[10 pt] %You can adjust how far below the table the text should appear
    % Coefficienti Kendall's $\tau$ ottenuti dai diversi algoritmi sulle reti DLN generate.
    % Per ogni rete sono evidenziati i punteggi ottenuti dai 3 algoritmi migliori.
    % \begin{tabular} 
    %     Should be a caption \\
    % \end{tabular}
\end{table}

% # dln
%          DLN_2_0.3_0.3_0.3
% ['mlPCI', 'laPCI', 'verPR_T']
%          DLN_2_0.3_0.3_0.8
% ['mlPCI', 'laPCI', 'alPCI']
%          DLN_2_0.3_0.8_0.3
% ['mlPCI', 'laPCI', 'alPCI']
%          DLN_2_0.3_0.8_0.8
% ['mlPCI', 'laPCI', 'alPCI']
%          DLN_2_0.8_0.3_0.3
% ['mlPCI', 'lsPCI', 'laPCI']
%          DLN_2_0.8_0.3_0.8
% ['mlPCI', 'laPCI', 'lsPCI']
%          DLN_2_0.8_0.8_0.3
% ['mlPCI', 'lsPCI', 'laPCI']
%          DLN_2_0.8_0.8_0.8
% ['mlPCI', 'lsPCI', 'laPCI']
%       addPR 0.4885 0.4844 0.4888 0.4860 0.4872 0.4874 0.4853 0.4848 
%     addPR_T 0.3649 0.4137 0.3601 0.4134 0.3744 0.4136 0.3755 0.4110 
%     aggCore 0.5575 0.5376 0.5552 0.5379 0.5596 0.5391 0.5578 0.5416 
%   aggCore_T 0.6063 0.5604 0.5946 0.5740 0.5718 0.5698 0.5823 0.5809 
%      aggDeg 0.6620 0.6402 0.6607 0.6401 0.6274 0.6201 0.6297 0.6179 
%       aggPR 0.5316 0.4958 0.5296 0.4972 0.5214 0.4879 0.5195 0.4869 
%     aggPR_T 0.6161 0.5469 0.6115 0.5461 0.5484 0.4965 0.5496 0.4930 
%       alPCI 0.6945 0.7036 0.7043 0.7109 0.6565 0.6694 0.6776 0.6800 
%       laPCI 0.7145 0.7157 0.7113 0.7158 0.6759 0.6910 0.6783 0.6868 
%       lsPCI 0.6313 0.6285 0.6354 0.6335 0.6840 0.6848 0.6879 0.6895 
%       mlPCI 0.7344 0.7406 0.7305 0.7427 0.6980 0.7114 0.7027 0.7095 
%   multiCore 0.3551 0.3574 0.3554 0.3576 0.3551 0.3549 0.3553 0.3550 
% multiCore_T 0.3646 0.3635 0.3647 0.3632 0.3648 0.3633 0.3649 0.3644 
%     sumCore 0.5225 0.5350 0.5259 0.5346 0.5279 0.5350 0.5259 0.5334 
%   sumCore_T 0.4377 0.4813 0.4340 0.4824 0.4468 0.4816 0.4474 0.4804 
%       verBC 0.6586 0.5605 0.6610 0.5639 0.6275 0.5404 0.6309 0.5441 
%       verPR 0.5275 0.5141 0.5261 0.5136 0.5195 0.5078 0.5162 0.5064 
%     verPR_T 0.6985 0.6547 0.6934 0.6535 0.6645 0.6554 0.6647 0.6364



% # dln
%          DLN_2_0.3_0.3_0.3
% ['mlPCI', 'laPCI', 'alPCI']
%          DLN_2_0.3_0.3_0.8
% ['mlPCI', 'laPCI', 'alPCI']
%          DLN_2_0.3_0.8_0.3
% ['mlPCI', 'laPCI', 'alPCI']
%          DLN_2_0.3_0.8_0.8
% ['mlPCI', 'laPCI', 'alPCI']
%          DLN_2_0.8_0.3_0.3
% ['mlPCI', 'lsPCI', 'laPCI']
%          DLN_2_0.8_0.3_0.8
% ['mlPCI', 'laPCI', 'lsPCI']
%          DLN_2_0.8_0.8_0.3
% ['mlPCI', 'lsPCI', 'laPCI']
%          DLN_2_0.8_0.8_0.8
% ['mlPCI', 'lsPCI', 'laPCI']
%       addPR 0.4885 0.4844 0.4888 0.4860 0.4872 0.4874 0.4853 0.4848 
%     aggCore 0.5575 0.5376 0.5552 0.5379 0.5596 0.5391 0.5578 0.5416 
%      aggDeg 0.6620 0.6402 0.6607 0.6401 0.6274 0.6201 0.6297 0.6179 
%       aggPR 0.5316 0.4958 0.5296 0.4972 0.5214 0.4879 0.5195 0.4869 
%       alPCI 0.6945 0.7036 0.7043 0.7109 0.6565 0.6694 0.6776 0.6800 
%       laPCI 0.7145 0.7157 0.7113 0.7158 0.6759 0.6910 0.6783 0.6868 
%       lsPCI 0.6313 0.6285 0.6354 0.6335 0.6840 0.6848 0.6879 0.6895 
%       mlPCI 0.7344 0.7406 0.7305 0.7427 0.6980 0.7114 0.7027 0.7095 
%   multiCore 0.3551 0.3574 0.3554 0.3576 0.3551 0.3549 0.3553 0.3550 
%     sumCore 0.5225 0.5350 0.5259 0.5346 0.5279 0.5350 0.5259 0.5334 
%       verBC 0.6586 0.5605 0.6610 0.5639 0.6275 0.5404 0.6309 0.5441 
%       verPR 0.5275 0.5141 0.5261 0.5136 0.5195 0.5078 0.5162 0.5064
\begin{table}[!htbp]
  \caption{Kendall's $\tau$ di algoritmi applicati al grafo trasposto in reti SLN}
  \label{tab:tauslnT}
    \centering
    \begin{tabular}{lrrrrrrrr}
        \toprule
          & SLN$_{\begin{subarray}{l} 0.3\text{,} \\ 0.3\text{,} \\ 0.3 \end{subarray}}$ 
          & SLN$_{\begin{subarray}{l} 0.3\text{,} \\ 0.3\text{,} \\ 0.8 \end{subarray}}$ 
          & SLN$_{\begin{subarray}{l} 0.3\text{,} \\ 0.8\text{,} \\ 0.3 \end{subarray}}$ 
          & SLN$_{\begin{subarray}{l} 0.3\text{,} \\ 0.8\text{,} \\ 0.8 \end{subarray}}$ 
          & SLN$_{\begin{subarray}{l} 0.8\text{,} \\ 0.3\text{,} \\ 0.3 \end{subarray}}$ 
          & SLN$_{\begin{subarray}{l} 0.8\text{,} \\ 0.3\text{,} \\ 0.8 \end{subarray}}$ 
          & SLN$_{\begin{subarray}{l} 0.8\text{,} \\ 0.8\text{,} \\ 0.3 \end{subarray}}$ 
          & SLN$_{\begin{subarray}{l} 0.8\text{,} \\ 0.8\text{,} \\ 0.8 \end{subarray}}$ \\
        \midrule
          addPR$^T$ &    {\num{ 0.4674}} &   {\num{ 0.4831}} &   {\num{ 0.4714}} &   {\num{ 0.4829}} &   {\num{ 0.4675}} &   {\num{ 0.4824}} &   {\num{ 0.4676}} &   {\num{ 0.4849}} \\
        aggCore$^T$ &    {\num{ 0.4608}} &   {\num{ 0.4225}} &   {\num{ 0.4601}} &   {\num{ 0.4300}} &   {\num{ 0.5110}} &   {\num{ 0.4362}} &   {\num{ 0.4811}} &   {\num{ 0.4663}} \\
          aggPR$^T$ &  \3{\num{ 0.5900}} & \3{\num{ 0.5700}} & \3{\num{ 0.5822}} & \3{\num{ 0.5625}} & \3{\num{ 0.5814}} & \3{\num{ 0.5413}} & \3{\num{ 0.5815}} & \3{\num{ 0.5362}} \\
      multiCore$^T$ &    {\num{ 0.4138}} &   {\num{ 0.3830}} &   {\num{ 0.4135}} &   {\num{ 0.4085}} &   {\num{ 0.4137}} &   {\num{ 0.4084}} &   {\num{ 0.4139}} &   {\num{ 0.4068}} \\
        sumCore$^T$ &  \1{\num{ 0.6339}} & \1{\num{ 0.6559}} & \1{\num{ 0.6382}} & \1{\num{ 0.6565}} & \2{\num{ 0.6296}} & \1{\num{ 0.6545}} & \2{\num{ 0.6316}} & \1{\num{ 0.6566}} \\
          verPR$^T$ &  \2{\num{ 0.6210}} & \2{\num{ 0.6245}} & \2{\num{ 0.6252}} & \2{\num{ 0.6251}} & \1{\num{ 0.6323}} & \2{\num{ 0.6240}} & \1{\num{ 0.6386}} & \2{\num{ 0.6207}} \\
        \bottomrule

    \end{tabular}
\end{table}

% # sln
%          SLN_2_0.3_0.3_0.3
% ['alPCI', 'sumCore_T', 'mlPCI']
%          SLN_2_0.3_0.3_0.8
% ['sumCore_T', 'verPR_T', 'alPCI']
%          SLN_2_0.3_0.8_0.3
% ['sumCore_T', 'alPCI', 'verPR_T']
%          SLN_2_0.3_0.8_0.8
% ['sumCore_T', 'verPR_T', 'alPCI']
%          SLN_2_0.8_0.3_0.3
% ['laPCI', 'alPCI', 'mlPCI']
%          SLN_2_0.8_0.3_0.8
% ['sumCore_T', 'alPCI', 'verPR_T']
%          SLN_2_0.8_0.8_0.3
% ['laPCI', 'mlPCI', 'alPCI']
%          SLN_2_0.8_0.8_0.8
% ['sumCore_T', 'verPR_T', 'alPCI']
%       addPR 0.4225 0.4293 0.4216 0.4295 0.4198 0.4309 0.4178 0.4287 
%     addPR_T 0.4674 0.4831 0.4714 0.4829 0.4675 0.4824 0.4676 0.4849 
%     aggCore 0.4066 0.3684 0.4158 0.3731 0.4010 0.3795 0.4117 0.3841 
%   aggCore_T 0.4608 0.4225 0.4601 0.4300 0.5110 0.4362 0.4811 0.4663 
%      aggDeg 0.6073 0.5520 0.6009 0.5460 0.6147 0.5490 0.6134 0.5412 
%       aggPR 0.4275 0.3619 0.4313 0.3695 0.4229 0.3713 0.4294 0.3779 
%     aggPR_T 0.5900 0.5700 0.5822 0.5625 0.5814 0.5413 0.5815 0.5362 
%       alPCI 0.6550 0.6152 0.6379 0.5971 0.6675 0.6300 0.6555 0.6203 
%       laPCI 0.6243 0.5761 0.6195 0.5687 0.6685 0.6208 0.6662 0.6092 
%       lsPCI 0.5901 0.5681 0.5728 0.5479 0.5962 0.5678 0.5897 0.5584 
%       mlPCI 0.6265 0.5810 0.6222 0.5771 0.6660 0.6169 0.6627 0.6057 
%   multiCore 0.3481 0.3560 0.3498 0.3576 0.3506 0.3535 0.3472 0.3513 
% multiCore_T 0.4138 0.3830 0.4135 0.4085 0.4137 0.4084 0.4139 0.4068 
%     sumCore 0.4209 0.4288 0.4211 0.4273 0.4164 0.4291 0.4171 0.4267 
%   sumCore_T 0.6339 0.6559 0.6382 0.6565 0.6296 0.6545 0.6316 0.6566 
%       verBC 0.5589 0.4528 0.5480 0.4506 0.5868 0.4842 0.5739 0.4796 
%       verPR 0.3650 0.3479 0.3697 0.3579 0.3200 0.3297 0.3317 0.3374 
%     verPR_T 0.6210 0.6245 0.6252 0.6251 0.6323 0.6240 0.6386 0.6207 

% # sln
%          SLN_2_0.3_0.3_0.3
% ['alPCI', 'mlPCI', 'laPCI']
%          SLN_2_0.3_0.3_0.8
% ['alPCI', 'mlPCI', 'laPCI']
%          SLN_2_0.3_0.8_0.3
% ['alPCI', 'mlPCI', 'laPCI']
%          SLN_2_0.3_0.8_0.8
% ['alPCI', 'mlPCI', 'laPCI']
%          SLN_2_0.8_0.3_0.3
% ['laPCI', 'alPCI', 'mlPCI']
%          SLN_2_0.8_0.3_0.8
% ['alPCI', 'laPCI', 'mlPCI']
%          SLN_2_0.8_0.8_0.3
% ['laPCI', 'mlPCI', 'alPCI']
%          SLN_2_0.8_0.8_0.8
% ['alPCI', 'laPCI', 'mlPCI']
%       addPR 0.4225 0.4293 0.4216 0.4295 0.4198 0.4309 0.4178 0.4287 
%     aggCore 0.4066 0.3684 0.4158 0.3731 0.4010 0.3795 0.4117 0.3841 
%      aggDeg 0.6073 0.5520 0.6009 0.5460 0.6147 0.5490 0.6134 0.5412 
%       aggPR 0.4275 0.3619 0.4313 0.3695 0.4229 0.3713 0.4294 0.3779 
%       alPCI 0.6550 0.6152 0.6379 0.5971 0.6675 0.6300 0.6555 0.6203 
%       laPCI 0.6243 0.5761 0.6195 0.5687 0.6685 0.6208 0.6662 0.6092 
%       lsPCI 0.5901 0.5681 0.5728 0.5479 0.5962 0.5678 0.5897 0.5584 
%       mlPCI 0.6265 0.5810 0.6222 0.5771 0.6660 0.6169 0.6627 0.6057 
%   multiCore 0.3481 0.3560 0.3498 0.3576 0.3506 0.3535 0.3472 0.3513 
%     sumCore 0.4209 0.4288 0.4211 0.4273 0.4164 0.4291 0.4171 0.4267 
%       verBC 0.5589 0.4528 0.5480 0.4506 0.5868 0.4842 0.5739 0.4796 
%       verPR 0.3650 0.3479 0.3697 0.3579 0.3200 0.3297 0.3317 0.3374 

\begin{table}[!htbp]
    \caption{Kendall's $\tau$ di algoritmi applicati alla rete trasposta in reti multiplex}
    \label{tab:taumuxT}
    \centering
    \begin{tabular}{lrrrrrr}
        \toprule
          & Drosophila & Homo & MA2013 & NYCM2014 
          & SacchCere & SacchPomb \\
        \midrule
          addPR$^T$ & \Fst{\num{ 0.6497}} &     {\num{ 0.5770}} & \Trd{\num{ 0.3193}} &     {\num{ 0.4235}} &     {\num{ 0.5356}} &     {\num{ 0.6701}} \\
        aggCore$^T$ & \Trd{\num{ 0.5686}} & \Fst{\num{ 0.7138}} & \Snd{\num{ 0.4756}} & \Fst{\num{ 0.5475}} & \Snd{\num{ 0.7073}} & \Fst{\num{ 0.7640}} \\
          aggPR$^T$ & \Snd{\num{ 0.6441}} & \Snd{\num{ 0.7033}} &     {\num{ 0.3156}} & \Trd{\num{ 0.4263}} & \Fst{\num{ 0.7202}} & \Trd{\num{ 0.7435}} \\
      multiCore$^T$ &     {\num{ 0.0763}} &     {\num{ 0.2165}} &     {\num{ 0.1491}} &     {\num{ 0.0127}} &     {\num{ 0.2569}} &     {\num{ 0.2110}} \\
        sumCore$^T$ &     {\num{ 0.4377}} & \Trd{\num{ 0.6689}} & \Fst{\num{ 0.4791}} & \Snd{\num{ 0.5380}} & \Trd{\num{ 0.6899}} & \Snd{\num{ 0.7464}} \\
          verPR$^T$ &     {\num{ 0.5280}} &     {\num{ 0.5797}} &     {\num{ 0.2286}} &     {\num{ 0.3878}} &     {\num{ 0.6370}} &     {\num{ 0.6799}} \\
        \bottomrule
    \end{tabular}
\end{table}


% # multiplex
%          drosophila_genetic_multiplex
% ['aggDeg', 'mlPCI', 'laPCI']
%          homo_genetic_multiplex
% ['mlPCI', 'aggDeg', 'laPCI']
%          MoscowAthletics2013_multiplex
% ['aggDeg', 'laPCI', 'mlPCI']
%          NYClimateMarch2014_multiplex
% ['laPCI', 'aggDeg', 'mlPCI']
%          sacchcere_genetic_multiplex
% ['mlPCI', 'laPCI', 'aggDeg']
%          sacchpomb_genetic_multiplex
% ['mlPCI', 'aggDeg', 'laPCI']
%       addPR 0.0398 0.3424 0.0059 -0.1313 0.3300 0.3410 
%     aggCore 0.0646 0.4132 0.0572 -0.0453 0.4384 0.1046 
%      aggDeg 0.7355 0.7096 0.5711 0.6150 0.6886 0.7656 
%       aggPR 0.0417 0.3857 0.0164 -0.0771 0.3944 0.2562 
%       alPCI 0.3682 0.1588 0.0493 0.1124 0.0455 0.4502 
%       laPCI 0.6040 0.6859 0.5534 0.6178 0.6980 0.6853 
%       lsPCI 0.0000 0.0000 0.0000 0.0000 0.0000 0.0000 
%       mlPCI 0.6947 0.7191 0.5532 0.6028 0.7073 0.7729 
%   multiCore 0.0497 0.1962 0.0428 0.0230 0.1925 -0.0170 
%     sumCore 0.0658 0.4575 0.0616 -0.0487 0.4546 0.1000 
%       verBC 0.5243 0.5653 0.1726 0.2045 0.5666 0.6550 
%       verPR -0.3505 -0.0606 -0.2707 -0.4107 0.0004 -0.4201 



% # multiplex
%          drosophila_genetic_multiplex
% ['aggDeg', 'mlPCI', 'addPR_T']
%          homo_genetic_multiplex
% ['mlPCI', 'aggCore_T', 'aggDeg']
%          MoscowAthletics2013_multiplex
% ['aggDeg', 'laPCI', 'mlPCI']
%          NYClimateMarch2014_multiplex
% ['laPCI', 'aggDeg', 'mlPCI']
%          sacchcere_genetic_multiplex
% ['aggPR_T', 'aggCore_T', 'mlPCI']
%          sacchpomb_genetic_multiplex
% ['mlPCI', 'aggDeg', 'aggCore_T']
%       addPR 0.0398 0.3424 0.0059 -0.1313 0.3300 0.3410 
%     addPR_T 0.6497 0.5770 0.3193 0.4235 0.5356 0.6701 
%     aggCore 0.0646 0.4132 0.0572 -0.0453 0.4384 0.1046 
%   aggCore_T 0.5686 0.7138 0.4756 0.5475 0.7073 0.7640 
%      aggDeg 0.7355 0.7096 0.5711 0.6150 0.6886 0.7656 
%       aggPR 0.0417 0.3857 0.0164 -0.0771 0.3944 0.2562 
%     aggPR_T 0.6441 0.7033 0.3156 0.4263 0.7202 0.7435 
%       alPCI 0.3682 0.1588 0.0493 0.1124 0.0455 0.4502 
%       laPCI 0.6040 0.6859 0.5534 0.6178 0.6980 0.6853 
%       lsPCI 0.0000 0.0000 0.0000 0.0000 0.0000 0.0000 
%       mlPCI 0.6947 0.7191 0.5532 0.6028 0.7073 0.7729 
%   multiCore 0.0497 0.1962 0.0428 0.0230 0.1925 -0.0170 
% multiCore_T 0.0763 0.2165 0.1491 0.0127 0.2569 0.2110 
%     sumCore 0.0658 0.4575 0.0616 -0.0487 0.4546 0.1000 
%   sumCore_T 0.4377 0.6689 0.4791 0.5380 0.6899 0.7464 
%       verBC 0.5243 0.5653 0.1726 0.2045 0.5666 0.6550 
%       verPR -0.3505 -0.0606 -0.2707 -0.4107 0.0004 -0.4201 
%     verPR_T 0.5280 0.5797 0.2286 0.3878 0.6370 0.6799 

