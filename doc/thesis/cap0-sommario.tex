\setlength{\parskip}{1em}

\chapter*{Sommario} % senza numerazione
\label{sommario}

\addcontentsline{toc}{chapter}{Sommario} % da aggiungere comunque all'indice

Diversi sistemi reali possono essere modellati tramite una rete, in cui
le entità, o \nodi, sono legate se tra loro c'è una relazione.
Attualmente, nella maggior parte dei casi una rete è rappresentata utilizzando 
un \grafo, dove è previsto un unico tipo di collegamento tra le entità.
In molti casi, però, le relazioni tra i nodi possono essere di varia natura. 
Questo tipo di sistema può essere rappresentato utilizzando una 
\muln.

Lo scopo di questa tesi è l'individuazione degli \infsp\ in una multilayer network, 
ovvero di quei nodi che durante un processo di diffusione di 
un'informazione nella rete permettono di raggiungere la porzione di rete maggiore. 
Questi nodi sono particolarmente utili in quanto possono essere sfruttati per 
massimizzare la diffusione dell'informazione o, al contrario, 
isolati se si vuole che essa resti il più possibile confinata.

% \\[1em]
Tali nodi sono facilmente identificabili simulando i processi di diffusione, 
che però risultano computazionalmente molto costosi, perciò questo metodo non 
è applicabile a reti molto grandi.
Pertanto, l'obiettivo è stato quello di verificare se esiste un modo efficace per 
individuarli a partire dalla topologia della rete, riducendo così la complessità computazionale 
e di conseguenza il tempo richiesto.
% \\[1em]
Una fase di ricerca iniziale ha permesso di raccogliere, studiare 
e successivamente implementare diversi 
algoritmi di centralità già utilizzati per risolvere questo problema, 
sia in grafi che in multilayer network.

Questi algoritmi sono stati applicati a diversi tipi di reti multilayer:
reti multiplex estratte a partire da dataset biologici e 
sociali e reti multilayer generate casualmente 
a partire da grafi derivati da applicazioni peer-to-peer.

I valori di centralità dei nodi restituiti da ogni algoritmo sono stati 
confrontati con quelli reali, ottenuti simulando dei processi di diffusione
nelle varie reti. In questo modo è stato possibile assegnare ad ogni algoritmo
un punteggio ottenuto su ciascuna rete e individuare così quali algoritmi
possono essere utilizzati per trovare gli influential spreaders in una multilayer network.
Due degli algoritmi testati hanno ottenuto buone performance su tutte le
reti analizzate, mentre le prestazioni degli altri sono state scarse o altalenanti.

% Il sommario dell’elaborato consiste al massimo di 3 pagine e deve contenere le seguenti informazioni:
% \begin{itemize}
%   \item contesto e motivazioni
%   \item breve riassunto del problema affrontato
%   \item tecniche utilizzate e/o sviluppate
%   \item risultati raggiunti, sottolineando il contributo personale del laureando/a
% \end{itemize}



