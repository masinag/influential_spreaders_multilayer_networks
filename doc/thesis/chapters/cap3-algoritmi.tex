\chapter{Algoritmi utilizzati}
Dopo una fase di ricerca, sono stati selezionati ed implementati alcuni algoritmi di centralità solitamente usati per 
l'individuazione degli influential spreaders. Tali algoritmi assegnano ad ogni nodo un valore proporzionale alla sua centralità 
secondo metriche differenti e quindi all'influenza esercitata da tale nodo sulla rete.

Questi possono essere divisi in tre categorie:
\begin{itemize}
\item algoritmi applicati al grafo aggregato;
\item algoritmi applicati separatamente ad ogni layer;
\item algoritmi applicati all'intera struttura multilayer.
\end{itemize}

\section{Algoritmi applicati al grafo aggregato}
Diverse misure di centralità sono utilizzate per identificare gli influential spreaders in grafi 
`classici', detti anche \textit{monoplex} siccome hanno un solo layer~\cite{basaras:infsp}\cite{kitsak:infsp}\cite{pei:infsp}. 
Essendo definite per grafi ad un solo layer, sono state applicate al grafo aggregato
derivato da ogni multilayer network. 
Le misure utilizzate sono state \emph{\PageRank}~\cite{page:pagerank}, 
\emph{\kcore}~\cite{batagelj:kcore} e 
\emph{Degree Centrality}.

% PAGE RANK

\subsection{\PageRank~(aggPR)}
\label{alg:pagerank}
\begin{definizione}[\PageRank]
    \label{def:pagerank}
    Dato un grafo $G=(V, E)$, il valore di \emph{PageRank} di un nodo $v$ è definito come
    \begin{equation*}
        \mathit{PR}(v) = \alpha\sum_{w : v \in G.\mathit{adj}(w)}
        \frac{\mathit{PR}(w)}{G.\mathit{outdegree}(w)} + 
        (1-\alpha)\frac{1}{|V|}
    \end{equation*}
    con $0 \le \alpha \le 1$.
\end{definizione}

Questa misura di centralità fu originariamente definita per misurare l'importanza di una pagina web. 
Secondo questo criterio, una pagina è tanto più importante quante più pagine importanti hanno un link 
verso di essa. 
\PageRank\ può essere visto come un modello di comportamento di un utente che, trovandosi inizialmente
in una pagina casuale, continua a visitare i link che trova, ma occasionalmente riparte da un'altra 
pagina casuale.
Analogamente, questo algoritmo può essere utilizzato per misurare più genericamente l'influenza di un nodo 
in un grafo.
Il parametro $\alpha$, detto \emph{damping factor}, regola la probabilità che si continui la navigazione dei link
ed è stato impostato a \num{0.85}.

Questa definizione ricorsiva è stata così calcolata in modo iterativo: 

% \begin{equation*}
%     \begin{split}
% PR_0(v)&= PR(v) \text{ in } X_1 \\
% addPR_l(v)&= \alpha \sum_{w : v \in X_l.adj(w)} 
%         \frac{addPR_l(w)}{X_l.outdegree(w)} + 
%         (1-\alpha)\frac{addPR_{l-1}(v)}{|V| \langle addPR_{l-1} \rangle}
%     \end{split}
% \end{equation*}
\begin{equation*}
    \begin{split}
\mathit{PR}_0(v)&= \frac{1}{n}\\
\mathit{PR}_i(v)&= \alpha \sum_{w : v \in G.\mathit{adj}(w)} 
\frac{\mathit{PR}_{i-1}(w)}{G.\mathit{outdegree(}w)} + 
(1-\alpha)\frac{1}{|V|}
    \end{split}
\end{equation*}
finché $||\mathit{PR}_{i} - \mathit{PR}_{i-1}||_1 < \epsilon$. Secondo Page et al., il numero di iterazioni 
è proporzionale a $\log{|V|}$~\cite{page:pagerank}. Dunque, la complessità dell'algoritmo è $O((|V| + |E|)\log{|V|})$.

% K-CORE

\subsection{\emph{k}-core~(aggCore)}
\begin{definizione}[\kcore]
    \label{def:kcore}
    Dato un grafo $G=(V, E)$, un \emph{k-core} di $G$ è un sottografo $G'=(V',E')$
    tale che 
    \begin{equation*}
        G'.\mathit{indegree}(v) \geq k \quad \forall v \in V'
    \end{equation*}
\end{definizione}
Un nodo appartenente ad un \kcore\ con $k$ elevato è considerato un nodo centrale 
nella rete, e quindi potenzialmente classificato tra gli influential spreaders.
Questa definizione può essere calcolata con complessità $O(|V| + |E|)$~\cite{batagelj:kcore}.

\subsection{Degree Centrality~(aggDeg)}
Questa misura di centralità assegna ad ogni nodo un valore pari al suo $\mathit{outdegree}$, 
per cui un nodo con tanti vicini si assume possa avere una certa rilevanza in un 
processo di diffusione.
Può essere calcolata con complessità $O(|V|)$.

\section{Algoritmi applicati ad ogni layer}
Alcune delle misure sono state estese alle reti multilayer calcolando la centralità
dei nodi in ogni layer separatamente e poi sommando i punteggi ottenuti in ogni layer.
In questa categoria rientrano \emph{\addPageRank}~\cite{halu:addpagerank} 
e \emph{\sumCore}~\cite{basaras:infspmul}.

\subsection{Additive PageRank~(addPR)}
\begin{definizione}[\addPageRank]
    Data una multilayer network $\mathcal{M}=(\mathcal{G}, \mathcal{E})$ 
    ed un ordinamento dei layer $(X_1, \dots, X_{L})$, l'\emph{additive PageRank}
    di un nodo $v \in V$ è il valore 
    $\mathit{addPR}(v) = \mathit{addPR}_L(v)$, dove:

    \begin{equation*}
        \begin{split}
\mathit{addPR}_1(v)&= \mathit{PR}(v) \text{ in } X_1 \\
\mathit{addPR}_l(v)&= \alpha \sum_{w : v \in X_l.\mathit{adj}(w)} 
            \frac{\mathit{addPR}_l(w)}{X_l.\mathit{outdegree}(w)} + 
            (1-\alpha)\frac{\mathit{addPR}_{l-1}(v)}{|V| \langle \mathit{addPR}_{l-1} \rangle}
        \end{split}
    \end{equation*}

\end{definizione}

Questa estensione dell'algoritmo di \PageRank\ alle reti multilayer richiede un ordinamento
dei layer. I layer sono stati ordinati in modo decrescente per autovalore di modulo massimo della rispettiva matrice di adiacenza~\cite{basaras:infspmul}, 
guidati dal fatto che autovalore più grande significa maggiore capacità di diffusione~\cite{wang:eigenv}.
Infatti, per questa definizione i layer ultimi nell'ordinamento incidono maggiormente nel calcolo del valore 
finale; inoltre, nei layer con capacità di diffusione molto elevata non è facile riconoscere gli influential spreaders, in quanto
si riesce a raggiungere una vasta porzione di rete indipendentemente dal nodo di partenza.  

Come algoritmo, è stato utilizzato un approccio analogo a quello della versione classica di 
\PageRank\ descritto nella Sezione~\vref{alg:pagerank}.
Poiché è necessario calcolare il \PageRank\ su ogni layer, il costo dell'algoritmo è 
$O(\sum_{i}((|V_i| + |E_i|)\log{|V_i|}))$, assumendo che i layer siano già ordinati.

\subsection{Sum \emph{k}-core~(sumCore)}
Questa misura è stata ottenuta calcolando il valore di \kcore\ di ogni nodo in 
ogni layer, quindi sommando i valori ottenuti da un nodo in tutti i layer in cui è presente.
Pertanto, la complessità del calcolo è pari a $O(\sum_{i}(|V_i| + |E_i|)$.

\section{Algoritmi applicati all'intera struttura multilayer}
Diverse misure definite originariamente per grafi monoplex sono state estese a 
reti multilayer. Queste, al contrario di quelle presentate nelle sezioni precedenti,
tengono conto della struttura a più livelli della rete ed in particolare delle inter-connessioni.
Quelle implementate sono 
\emph{\verPageRank}~\cite{dedomenico:versatile},
\emph{\verBetweennessCentrality}~\cite{dedomenico:versatile}\cite{dedomenico:verbetw},
\emph{\multiCore}~\cite{azimi:multikcore} e le diverse varianti di \emph{\PCI}: 
\emph{\mlPCI},
\emph{\laPCI},
\emph{\alPCI},
\emph{\lsPCI}~\cite{basaras:infspmul}.

\subsection{Versatility PageRank~(verPR)}
\begin{definizione}[\verPageRank]
    Data una multilayer network $\mathcal{M}=(\mathcal{G}, \mathcal{E})$,
    si definisce il valore di \mbox{\emph{\verPageRank}} di un nodo $v \in V_l$ in un layer $l$
    come:

    \begin{equation*}
        \mathit{verPR}(v, l) = 
        \alpha \biggl( 
            \sum_{w : v \in G_l.\mathit{adj}(w)} \frac{\mathit{verPR}(w, l)}{\mathcal{M}.\mathit{outdegree}(w, l)} +
            \sum_{j \neq l}^{}\sum_{w : (w, v) \in E_{jl}} \frac{\mathit{verPR}(w, j)}{\mathcal{M}.\mathit{outdegree}(w, j)} 
        \biggr) + (1-\alpha)\frac{1}{N}
    \end{equation*}
    dove $\mathcal{M}.\mathit{outdegree}(w, l)$ è la somma del numero di archi e di inter-connessioni uscenti da $w$ nel layer $l$.

\end{definizione}

La definizione originale utilizzava un tensore 4-dimensionale per 
rappresentare la rete~\cite{dedomenico:versatile}. Qui è stata generalizzata, anche prevedendo che un nodo possa non comparire 
in tutti i layer.
Questa definizione è stata calcolata in un modo analogo a quello descritto nella Sezione \ref{alg:pagerank},
dunque la complessità è $O((N + E)\log{N})$.

\subsection{Versatility Betweenness Centrality~(verBC)}

\begin{definizione}[\verBetweennessCentrality]
    Data una multilayer network $\mathcal{M}=(\mathcal{G}, \mathcal{E})$, la 
    \emph{\verBetweennessCentrality} di un nodo $v$ è il valore 
    \begin{equation*}
        \mathit{verBC}(v) = \sum_{s, t \in V} \frac{\sigma_{st}(v)}{\sigma_{st}}
    \end{equation*}
    dove $\sigma_{st}$ è il numero di percorsi minimi tra il nodo $s$ ed il nodo $t$
    indipendentemente dal layer, e $\sigma_{st}(v)$ è il numero di questi che passa 
    per il nodo $v$ in qualche layer.
\end{definizione}

L'algoritmo descritto da Ribalta et al. permette di calcolare questa definizione con complessità
$O\left(|V|\left(N+E\right)\right)$~\cite{dedomenico:verbetw}.

\subsection{Multiplex \emph{k}-core~(multiCore)}

\begin{definizione}[\multiCore]
    Data una multilayer network $\mathcal{M}=(\mathcal{G}, \mathcal{E})$, il \emph{\multiCore} è
    il più grande sottografo per cui ogni nodo è raggiunto in ogni layer da almeno $k$ archi o
    inter-connessioni.
\end{definizione}

Questa misura è stata calcolata estendendo l'algoritmo descritto da Batagelj e Zaversnik per calcolare 
il $k$-core di un grafo per adattarlo a questa definizione, ottenendo una complessità di $O(N + E)$~\cite{batagelj:kcore}.

\subsection{\PCI~(PCI)}
\begin{definizione}
    Data una multilayer network $\mathcal{M}=(\mathcal{G}, \mathcal{E})$, 
    l'indice $\mathit{mlPCI}_n$ di un nodo $v$ in un layer $l$ 
    è il massimo numero $k$ tale che esistono almeno $k$ vicini di $v$ nel layer $l$ in $\mathcal{M}$
    con numero di vicini in almeno $n$ layer maggiore o uguale a $k$.
\end{definizione}

Da questa definizione si ricavano le seguenti:

\begin{definizione}[\mlPCI]
    L'indice $\mathit{mlPCI}$ di un nodo $v$ in un layer $l$ è definito come 
    \begin{equation*}
        \mathit{mlPCI}(v, l) = \sum_{n=1}^L \mathit{mlPCI}_n(v, l)
    \end{equation*}
\end{definizione}

\begin{definizione}[\alPCI]
    L'indice $\mathit{alPCI}$ di un nodo $v$ in un layer $l$ è definito come 
    \begin{equation*}
        \mathit{alPCI}(v, l) = \mathit{mlPCI}_L(v, l)
    \end{equation*}
\end{definizione}


\begin{definizione}[\lsPCI]
    L'indice $\mathit{lsPCI}$ di un nodo $v$ in un layer $l$ è il massimo numero $k$ 
    tale che esistono almeno $k$ vicini di $v$ nel layer $l$ in $\mathcal{M}$
    con numero di vicini in almeno $k$ layer maggiore o uguale a $k$.
\end{definizione}

\begin{definizione}[\laPCI]
    L'indice $\mathit{laPCI}$ di un nodo $v$ in un layer $l$ 
    è il massimo numero $k$ tale che esistono almeno $k$ vicini di $v$ nel layer 
    $l$ in $\mathcal{M}$
    con numero di vicini maggiore o uguale a $k$.
\end{definizione}



Il calcolo di queste definizioni è stato fatto utilizzando un algoritmo di complessità 
$O(NLt\log{t})$, dove $t = \max{\{\mathcal{M}.\mathit{outdegree}(v)\,|\,v \in V\}}$ 
