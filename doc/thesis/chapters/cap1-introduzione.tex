\chapter{Introduzione}

Una rete è una struttura in cui sono presenti diverse entità connesse tra loro da legami 
di qualche tipo. Molte realtà possono essere modellate utilizzando una rete:
% Una grande quantità di sistemi del mondo possono essere visti come una rete, ossia una struttura 
% in cui diverse entità sono legate tra loro tramite relazioni. 
% % Le reti permettono di rappresentare una grande quantità di sistemi in cui diverse entità 
% % sono legate tra loro mediante delle relazioni.
alcuni esempi sono le reti di interazione sociale, in cui le persone sono legate 
da relazioni di amicizia, di collaborazione o di contatto sui social network, 
le reti dei trasporti, come quelle ferroviaria e aerea
che collegano diverse città, e le reti biologiche, come quella alimentare o
quella delle interazioni tra molecole all'interno di un organismo.

Utilizzando questi modelli è particolarmente utile lo studio dei processi di diffusione
di un'informazione in una rete per comprendere le dinamiche di diversi scenari.
Tra questi, la diffusione di un virus in una società, di una notizia in una rete di contatti, 
di una fake news in un social network e l'accumulo di un ritardo nelle stazioni di una rete 
dei trasporti.

In questi processi, inizialmente un \emph{nodo} (così sono chiamate le entità all'interno di una rete) 
è `infetto'; in ogni instante un nodo infetto può `contagiare' con una certa probabilità i 
nodi con cui è in relazione, rendendoli a loro volta infetti.
Di particolare interesse è l'individuazione dei cosiddetti \infsp, ovvero dei nodi
che, se infettati, permettono di contagiare una grossa parte della rete.
Tali nodi, infatti, possono essere sfruttati per garantire un'efficace diffusione 
di un'informazione nella rete o, al contrario, per evitarne la propagazione.
Ad esempio, per diffondere una notizia importante in un social network sarà opportuno che 
questa venga pubblicata da un account molto seguito, mentre per fermare un'epidemia 
può essere utile isolare i luoghi o le persone che hanno un ruolo più centrale
nella società.


\section{Influential spreaders in un grafo}

Generalmente, una rete viene modellata con un \emph{grafo}, ossia una struttura in cui i 
\emph{nodi} sono connessi da un \emph{arco} se tra loro esiste una relazione.

Il modo più affidabile per l'individuazione degli influential spreaders in un grafo consiste nel 
simulare un processo di diffusione, detto anche \spproc, partendo da ogni nodo e 
classificare come nodi più 
influenti quelli da cui si riesce ad infettare la porzione di rete maggiore.
Tuttavia, poiché la trasmissione tra un nodo e un suo vicino è probabilistica,
occorrerebbe effettuare un gran numero di simulazioni prima di avere una stima accurata 
dell'influenza di un nodo sulla rete e ciò rende questo metodo computazionalmente 
molto costoso.

Un problema molto studiato è quello dell'individuazione degli influential spreaders tramite l'analisi della 
topologia delle rete.
Diversi studi mostrano come alcuni algoritmi di centralità possono fornire una buona indicazione di 
quali sono i nodi più influenti nella diffusione~\cite{basaras:infsp}\cite{kitsak:infsp}\cite{pei:infsp}.
Tra questi, alcuni richiedono una conoscenza globale della topologia della rete come
\emph{\PageRank}, \emph{Betweenness Centrality} e \emph{\kcore}, 
mentre per altri come \emph{Degree Centrality} o \emph{$\mu$-PCI} ogni nodo 
ha bisogno solo di una conoscenza locale.

\section{Influential spreaders in una multilayer network}

In molti degli scenari reali, tuttavia, i nodi sono potenzialmente collegati da diversi tipi di relazione, che non 
possono essere rappresentati in un singolo grafo. 
Tra questi troviamo la rete dei profili online collegati dalle loro interazioni
in diversi social network (follow, amicizie, condivisioni, ecc.) e la rete delle stazioni di 
una città connesse da tratte di diversi mezzi di trasporto (rete ferroviaria, rete degli autobus, 
piste ciclabili, ecc.).

Strutture che permettono di rappresentare queste realtà sono dette \muln, ossia reti in cui sono 
presenti differenti tipi di relazione tra i nodi. Fornendo un modello più fedele 
delle reti reali, la loro analisi può dare risultati più accurati rispetto 
a quelli ottenibili rappresentando la realtà con un grafo.

In questa tesi viene affrontato il problema dell'individuazione degli influential spreaders in 
una multilayer network, calcolando diverse misure di centralità e confrontando i 
risultati con quelli ottenuti simulando degli spreading process.
