\chapter{Dataset}
Queste misure sono state testate su diverse reti con caratteristiche diverse:
\begin{itemize}
    \item Multiplex network estratte da dataset biologici e sociali;
    \item Multilayer network generate a partire da reti di interazioni in Internet.
\end{itemize}

\section{Multiplex network}
Le reti multiplex utilizzate sono state ottenute a partire da reti multiplex biologiche e sociali,
estraendo da ognuna una sottorete in cui ogni nodo ha una controparte in tutti i layer.

\subsection{Descrizione dei dataset}
I dataset di partenza sono indicati in Tabella~\ref{tab:multiplex}.
Di seguito è riportata la descrizione di ognuno~\cite{data:multiplex}\cite{dedomenico:structred}\cite{stark:biogrid}:
\begin{description}
    \item[Drosophila]: multiplex network delle interazioni genetiche e proteiche di 
    Drosophila Melanogaster. I layer rappresentano:
    \begin{enumerate}
        \item Direct interaction;
        \item Suppressive genetic interaction defined by inequality;
        \item Additive genetic interaction defined by inequality;
        \item Physical association;
        \item Colocalization;
        \item Association;
        \item Synthetic genetic interaction defined by inequality.
    \end{enumerate} 
    
    \item[Homo]: multiplex network delle interazioni genetiche e proteiche di 
    Homo Sapiens. I layer rappresentano:
    \begin{enumerate}
        \item Direct interaction;
        \item Physical association;
        \item Suppressive genetic interaction defined by inequality;
        \item Association;
        \item Colocalization;
        \item Additive genetic interaction defined by inequality;
        \item Synthetic genetic interaction defined by inequality.
    \end{enumerate} 
    \item[MA2013]: multiplex network che rappresenta differenti tipi di relazioni sociali 
        tra utenti di Twitter durante i campionati del mondo di atletica leggera 
        di Mosca 2013.
        I layer rappresentano:
        \begin{enumerate}
            \item Retweet;
            \item Mentions;
            \item Replies.
        \end{enumerate}
    \item[NYCM2014]: multiplex network che rappresenta differenti tipi di relazioni sociali 
        tra utenti di Twitter durante la Marcia per il Clima svoltasi a New York nel 2014.
        I layer rappresentano:
        \begin{enumerate}
            \item Retweet;
            \item Mentions;
            \item Replies.
        \end{enumerate}
    \item[SacchCere]: multiplex network delle interazioni genetiche e proteiche di 
        Saccharomyces Cerevisiae. I layer rappresentano:
        \begin{enumerate}
            \item Physical association;
            \item Suppressive genetic interaction defined by inequality;
            \item Direct interaction;
            \item Synthetic genetic interaction defined by inequality;
            \item Association;
            \item Colocalization;
            \item Additive genetic interaction defined by inequality.
        \end{enumerate} 
    \item[SacchPomb]: multiplex network delle interazioni genetiche e proteiche di 
        Saccharomyces Pombe. I layer rappresentano:
        \begin{enumerate}
            \item Direct interaction;
            \item Colocalization;
            \item Physical association;
            \item Suppressive genetic interaction defined by inequality;
            \item Synthetic genetic interaction defined by inequality;
            \item Additive genetic interaction defined by inequality;
            \item Association.
        \end{enumerate} 
\end{description}

\subsection{Estrazione delle multiplex network}
L'obiettivo era quello di ottenere un dataset simile a quello utilizzato da Basaras et al.,
dove da ogni rete viene estratta una sottorete in cui ogni nodo è presente in tutti i layer~\cite{basaras:infspmul}.
Nell'articolo sono indicati per ogni rete i numeri $N$, $E$ ed $|L|$, rispettivamente di nodi, 
archi e layer estratti.

Per ogni rete, è stato individuato un sottoinsieme $L$ di $|L|$ layers con
numero di nodi con una controparte in ogni layer maggiore o uguale a $N$. 
Per tutte le reti, è  stato trovato un solo sottoinsieme 
che rispettasse questo criterio.

Dopo aver costruito una sottorete formata solo dai layer in $L$, sono stati rimossi i nodi non presenti
in tutti i layer e gli archi da e verso essi. Dopodiché sono stati rimossi tutti i nodi da cui non arrivava
né partiva più nessun arco. Questa procedura è stata applicata ricorsivamente alla sottorete ottenuta, 
finché non è stato più possibile rimuovere alcun nodo.

I risultati ottenuti sono stati molto simili a quelli voluti, come si può vedere nella Tabella \vref{tab:multiplex}.

In queste reti sono state aggiunte inter-connessioni tra le controparti di ogni nodo in ogni layer.


% reti multiplex 

\begin{table}
    \caption{Multiplex network estratte}
    \label{tab:multiplex}
    \centering
    \begin{tabular}{lrrrrrrrr}
        \toprule
              \multirow{2}*{Nome} 
            & \multicolumn{3}{c}{Nodi} 
            & \multicolumn{3}{c}{Archi} 
            & \multicolumn{2}{c}{Layers} \\
            \cmidrule(lr){2-4} \cmidrule(lr){5-7} \cmidrule(lr){8-9}                    
            
            
            & iniziali & obiettivo & estratti 
            & iniziali & obiettivo & estratti 
            & iniziali & estratti \\
            
            % Nodi iniziali & Nodi estratti in  & Nodi estratti & Archi iniziali & Archi estratti in  & Archi estratti & Layer iniziali & Layer estratti \\
        \midrule
            Drosophila          & \num{  8215} & \num{  1364} & \num{1375}
                                & \num{ 43366} & \num{  7267} & \num{7438} 
                                & \num{     7} & \num{     2} \\
            Homo                & \num{ 18222} & \num{  3859} & \num{3878}
                                & \num{170899} & \num{ 77483} & \num{78804} 
                                & \num{     7} & \num{     3} \\
            MA2013              & \num{ 88804} & \num{  4370} & \num{4377}
                                & \num{210250} & \num{ 33411} & \num{34589} 
                                & \num{     3} & \num{     3} \\
            NYCM2014            & \num{102439} & \num{  4150} & \num{4262}
                                & \num{353495} & \num{ 45334} & \num{47840} 
                                & \num{     3} & \num{     3} \\
            SacchCere           & \num{  6570} & \num{  3096} & \num{3121}
                                & \num{ 28275} & \num{185849} & \num{188182} 
                                & \num{     7} & \num{     5} \\
            SacchPomb           & \num{  4092} & \num{   875} & \num{878}
                                & \num{ 63676} & \num{ 18214} & \num{ 18308} 
                                & \num{     7} & \num{     3} \\
        \bottomrule
    \end{tabular}
%         
    
\end{table}

\section{Multilayer network}
Le reti multilayer sono state generate partendo da grafi ottenuti da 
diverse applicazioni peer-to-peer, generando poi casualmente delle inter-connessioni 
tra nodi di layer differenti.  

\subsection{Descrizione dei dataset}
I grafi utilizzati come layer sono indicati in Tabella~\vref{tab:multilayer}.
Di seguito è riportata la descrizione di ognuno~\cite{data:multilayer}:

\begin{description}
    \item[wiki-Vote]: grafo delle votazioni per le elezioni degli amministratori 
    di Wikipedia svoltesi fino al gennaio 2008.
    I nodi del grafo rappresentano utenti di Wikipedia, di cui la metà circa sono 
    amministratori mentre gli altri sono utenti ordinari, e un arco diretto dal nodo 
    $i$ al nodo $j$ indica che l'utente $i$ ha votato per l'utente $j$;

    \item[cit-HepTh]: grafo delle citazioni tra articoli sul sito arXiv
    pubblicati tra gennaio 1993 e aprile 2003.
    Se un paper $i$ cita un paper $j$, il grafo contiene un arco diretto 
    da $i$ a $j$. Non è contenuta alcuna informazione riguardo articoli non 
    presenti sul sito;

    \item[p2p-Gnutella04\textnormal{,} p2p-Gnutella05\textnormal{,} p2p-Gnutella06\textnormal{,} p2p-Gnutella08]: 
    grafi che rappresentano `istantanee' della rete di file sharing peer-to-peer Gnutella, nell'agosto 2002.
    In questi grafi i nodi rappresentano gli host della rete Gnutella e gli archi 
    rappresentano connessioni tra gli host.

\end{description}

% dataset multilayer
\begin{table}
    \caption{Dataset per la generazione di reti multilayer}
    \label{tab:multilayer}
    \centering
    \begin{tabular}{rlrr}
        \toprule
            Numero & Nome & Nodi & Archi \\
            % Nodi iniziali & Nodi estratti in  & Nodi estratti & Archi iniziali & Archi estratti in  & Archi estratti & Layer iniziali & Layer estratti \\
        \midrule
        1. & wiki-Vote & \num{7115} & \num{103689} \\
        2. & cit-HepTh & \num{27770} & \num{352807} \\
        3. & p2p-Gnutella04 & \num{10876} &  \num{39994} \\ 
        4. & p2p-Gnutella05  & \num{8846} & \num{31839} \\ 
        5. & p2p-Gnutella06  & \num{8717} & \num{31525} \\ 
        6. & p2p-Gnutella08 & \num{6301} & \num{20777} \\
        \bottomrule
    \end{tabular}
%         
\end{table}


\subsection{Generazione delle multilayer network}

Sono stati generati due tipi di reti~\cite{basaras:infspmul}:
\begin{itemize}
    \item \emph{Similar Layers Networks} (SLN), i cui layer sono i grafi [3-6] della  
            Tabella~\vref{tab:multilayer}. In queste reti i layer hanno tutti un numero simile 
            di nodi e archi;
    \item \emph{Different Layers Networks} (DLN), i cui layer sono i grafi [1-3] della 
        Tabella~\ref{tab:multilayer}. In queste reti i layer hanno numero di nodi e archi molto differenti.
\end{itemize}


Per prima cosa, sono state aggiunte inter-connessioni tra le controparti di ogni nodo in tutti 
i layer in cui esso è presente.
Quindi, sono state generate casualmente altre inter-connessioni impostando tre parametri:
\begin{enumerate}
    \item il numero di inter-connessioni uscenti da un nodo in un determinato layer;
    \item la frequenza con cui un layer viene scelto come layer di destinazione di 
        una connessione;
    \item la frequenza con cui un nodo in un certo layer viene scelto come nodo 
        di destinazione. 
\end{enumerate}
Ognuno di questi parametri è stato generato secondo una distribuzione che segue 
la legge di Zipf, per cui la frequenza di un certo valore è inversamente
proporzionale al valore stesso~\cite{zipf:humanb}. 
Tale distribuzione genera numeri tra 0 ed un valore massimo $m$ (escluso) con diverse probabilità,
secondo il valore di una variabile $s \in [0, 1]$
che regola il grado di `asimmetria' dei valori generati, in particolare:
\begin{itemize}
    \item per $s=0$ la probabilità $\mathcal{P}$ di ogni valore è la stessa;  
    \item per $s=1$ si ha $\mathcal{P}(0) = 2 \cdot \mathcal{P}(1) = 3 \cdot \mathcal{P}(2) =\dots = m \cdot \mathcal{P}(m-1)$. 
\end{itemize}

Le variabili che regolano i tre parametri sono chiamate rispettivamente
$s_{degree}$, $s_{layer}$, $s_{node}$. 
% La probabilità che un nodo abbia $k$ \interc\ è $\mathcal{P}_{degree}(k)$.
Per ognuna sono stati sperimentati i valori $0.3$ e $0.8$. 
Per quanto riguarda il numero di inter-connessioni di un nodo in un determinato layer
il valore massimo è stato impostato a $d \cdot \log_2{\sum_{i}{V_i}}$, con 
$d = 2$. Per gli altri parametri, invece, tutti i layer devono essere selezionabili
così come tutti i nodi all'interno di un layer.

L'Algoritmo~\vref{alg:generate} mostra come sono state generate le reti.
Nell'algoritmo, gli oggetti della classe \tzgen\ sono generatori di numeri 
casuali secondo una distribuzione che segue la legge di Zipf. È stato utilizzato 
un generatore per ognuno dei tre parametri:
\begin{enumerate}
    \item $\mathit{degree\_generator}$ genera numeri tra 0 e $\mathit{max\_interconnectons} - 1$
        con $s = s_{\mathit{degree}}$ che vengono utilizzati direttamente per definire il 
        numero di \interc\ uscenti da ogni nodo;
    \item $\mathit{layer\_generator}$ genera numeri tra 0 e $L - 1$, dove $L$ è il 
        numero di layers, con $s = s_{\mathit{layer}}$. Questi numeri vengono utilizzati 
        come indice per scegliere il layer di destinazione all'interno di una 
        permutazione dei layer $\mathit{perm\_l}$;
    \item $\mathit{node\_generators}$ è un vettore di generatori in cui 
        l'$i$-esimo elemento $\mathit{node\_generators}[i]$ genera numeri 
        tra 0 e $|V_i| - 1$, dove $|V_i|$ è il numero di nodi dell'$i$-esimo layer, 
        con $s = s_{\mathit{node}}$. Questi numeri vengono utilizzati 
        come indice per scegliere il nodo di destinazione all'interno 
        di una permutazione dei nodi del layer $i$ selezionato, 
        chiamata $\mathit{perm\_n}[i]$.
\end{enumerate}

Le reti così generate sono denominate con 
SLN{\Large$_{ s_{\mathit{degree}}\text{,}s_{\mathit{layer}}\text{,}s_{\mathit{node}}}$}
oppure
DLN{\Large$_{ s_{\mathit{degree}}\text{,}s_{\mathit{layer}}\text{,}s_{\mathit{node}}}$}


\SetStartEndCondition{ (}{)}{)}\SetAlgoBlockMarkers{}{\}}%
% \SetKwProg{Fn}{dsadsadas}{dsadad}
% \SetKwFunction{FRecurs}{generate\_multilayer}%
\SetKwFor{For}{for }{ \textbf{do}}{}{}%
\SetKwFor{If}{if }{ \textbf{then}}{}{}%
\SetKwFor{While}{while }{ \textbf{do}}{}{}%
% \SetKwIF{If}{ElseIf}{Else}{if}{\{}{elif}{else\{}{}%
% \SetKwFor{While}{while}{\{}{}%
% \SetKwRepeat{Repeat}{repeat\{}{until}%
% \AlgoDisplayBlockMarkers
% \SetAlgoNoLine%
\begin{algorithm}   
    \caption{Generazione multilayer networks}
    \label{alg:generate}
% \Fn{\FRecurs{asd}}
    % {
    \SetKwData{Left}{left}
    \SetKwData{This}{this}
    \SetKwData{Up}{up}
    \SetKwFunction{Union}{Union}
    \SetKwFunction{FindCompress}{FindCompress}
    
    \SetKwInOut{Input}{Input}
    \SetKwInOut{Output}{Output}
    \SetStartEndCondition{ }{}{}%
    \DontPrintSemicolon
    \Input{
        \textsc{Graph} $layers$\abracks \algcmnt{grafi usati come layers} \\ 
        \tint\ $perm\_l$\abracks \algcmnt{posizione di ogni layer nella permutazione}\\
        \tint\ $perm\_n$\abracks \abracks \algcmnt{posizione di un nodo nella permutazione di ogni layer}\\
        \tint\ $d $\\
        \tfloat\ $s\_degree$ \\
        \tfloat\ $s\_layer$ \\
        \tfloat\ $s\_node$ \\
    }
    
    \tint\ $total\_nodes = 0$\;
    \For{$i = 0$ \KwTo $layers.size()$} {
            $total\_nodes = total\_nodes + layers[i].nodes().size()$\;
    }
    \tint\ $max\_interconnections = d \cdot \lfloor\log_{2}{(total\_nodes)}\rfloor$\;
    \BlankLine
    
    \tcp{I due parametri della classe \textsc{ZipfGenerator} regolano rispettivamente \\
    il massimo della distribuzione e il grado di asimmetria}
    \tzgen\ $degree\_generator(max\_interconnections, s\_degree)$\;
    \tzgen\ $layer\_generator(layers.size(), s\_layer)$\;
    \tzgen\ $node\_generators[0\dots layers.size() - 1]$\;
    \For{$l = 0$ \KwTo $layers.size() - 1$}{
        $node\_generators[l] = $ \tzgen $(layers[l].size(), s\_degree)$\;
    }
    \BlankLine

    \textsc{MultilayerNetwork} $m$\;


    \For {$l = 0$ \KwTo $layers.size() - 1$}{
        \ForEach {$n \in layers[l]$}{
            \tcp*{intra-connessioni}
            \For {$v \in layers[l].adj(n)$}{
                $m.add\_edge(n, l, v, l)$\;
            }
            \tcp*{inter-connessioni tra un nodo e le sue controparti negli altri layer}
            \For {$j = 0$ \KwTo $layers.size() - 1$}{ 
                \If {$ j \neq l$ \textup{\textbf{and}} $n \in layers[j].nodes()$} {
                    $m.add\_edge(n, l, n, j)$\;
                }
            }
            \tcp*{Generazione inter-connessioni casuali}
            \tint\ $degree = degree\_generator.next()$\;
            \For {$i = 1$ \KwTo $degree$}{
                \tbool\ $added = \False$\;
                \While{\textup{\textbf{not}} $added$}{
                    \tint\ $l\_dest = l\_index[layer\_generator.next()]$\;
                    \While {$l\_dest == l$} {
                        $l\_dest = l\_index[layer\_generator.next()]$\;
                    }
                    \tint\ $n\_dest= n\_index[l\_dest][node\_generators[l\_dest].next()]$\;
                    \If{\textup{\textbf{not}} $m.has\_edge(n, l, n\_dest, l\_dest)$}{
                        $added = \True$\;
                        $m.add\_edge(n, l, n\_dest, l\_dest)$\;
                    }
                }
            }
        }
    }
\end{algorithm}
  
%   $\text{SLN}_d(s_{degree}$, $s_{layer}$, $s_{node})$
% oppure $\text{DLN}_d(s_{degree}$, $s_{layer}$, $s_{node})$.
