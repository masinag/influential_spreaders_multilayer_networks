\chapter{Dataset}
Queste misure sono state testate su diverse reti con caratteristiche diverse, 
seguendo una metodologia simile a \cite{basaras:infspmul}:
\begin{itemize}
    \item Multiplex Network estratte da dataset biologici e sociali;
    \item Multilayer Network generate a partire da reti di interazioni in Internet.
\end{itemize}

\section{Estrazione reti multiplex}
Le reti indicate in tabella \vref{tab:multiplex} sono state estratte a partire da reti biologiche e
sociali fornite da \cite{data:multiplex}.
L'obiettivo era quello di ottenere un dataset simile a quello utilizzato da \cite{basaras:infspmul},
dove per ogni rete sono indicati i numeri $N$, $E$ ed $|L|$, rispettivamente di nodi, archi e layer estratti.

Per ogni rete, è stato individuato un sottoinsieme $L$ di $|L|$ layer in cui il numero di nodi con una 
controparte in ogni layer fosse almeno $N$. Per tutte le reti, è  stato trovato un solo sottoinsieme 
che rispettasse questo criterio.

Dopo aver costruito una sotto-rete formata solo dai layer in $L$, sono stati rimossi i nodi non presenti
in tutti i layer e gli archi da e verso essi. Dopodichè sono stati rimossi tutti i nodi da cui non arrivava
nè partiva più nessun arco. Questa procedura è stata applicata ricorsivamente alla sotto-rete ottenuta, 
finchè non è stato più possibile rimuovere alcun nodo. 

I risultati ottenuti sono stati molto simili a quelli voluti, come si può vedere in tabella \vref{tab:multiplex}.

In queste reti sono state aggiunte \interc\ tra le controparti di ogni nodo in ogni layer.


% reti multiplex 

\begin{table}
    \caption{Multiplex network estratte}
    \label{tab:multiplex}
    \centering
    \begin{tabular}{lrrrrrrrr}
        \toprule
              \multirow{2}*{Nome} 
            & \multicolumn{3}{c}{Nodi} 
            & \multicolumn{3}{c}{Archi} 
            & \multicolumn{2}{c}{Layers} \\
            \cmidrule(lr){2-4} \cmidrule(lr){5-7} \cmidrule(lr){8-9}                    
            
            
            & iniziali & obiettivo & estratti 
            & iniziali & obiettivo & estratti 
            & iniziali & estratti \\
            
            % Nodi iniziali & Nodi estratti in  & Nodi estratti & Archi iniziali & Archi estratti in  & Archi estratti & Layer iniziali & Layer estratti \\
        \midrule
            Drosophila          & \num{  8215} & \num{  1364} & \num{1375}
                                & \num{ 43366} & \num{  7267} & \num{7438} 
                                & \num{     7} & \num{     2} \\
            Homo                & \num{ 18222} & \num{  3859} & \num{3878}
                                & \num{170899} & \num{ 77483} & \num{78804} 
                                & \num{     7} & \num{     3} \\
            MA2013              & \num{ 88804} & \num{  4370} & \num{4377}
                                & \num{210250} & \num{ 33411} & \num{34589} 
                                & \num{     3} & \num{     3} \\
            NYCM2014            & \num{102439} & \num{  4150} & \num{4262}
                                & \num{353495} & \num{ 45334} & \num{47840} 
                                & \num{     3} & \num{     3} \\
            SacchCere           & \num{  6570} & \num{  3096} & \num{3121}
                                & \num{ 28275} & \num{185849} & \num{188182} 
                                & \num{     7} & \num{     5} \\
            SacchPomb           & \num{  4092} & \num{   875} & \num{878}
                                & \num{ 63676} & \num{ 18214} & \num{ 18308} 
                                & \num{     7} & \num{     3} \\
        \bottomrule
    \end{tabular}
%         
    
\end{table}

\section{Generazione reti multilayer}
Sono state generate delle \muln\ partendo da grafi di interazioni su 
social network e su architetture peer-to-peer, in particolare quelli 
indicati in tabella \vref{tab:multilayer}, che si possono trovare in \cite{data:multilayer}.

Seguendo l'approccio di \cite{basaras:infspmul}, sono stati generati due tipi di reti:
\begin{itemize}
    \item Similar Layers Network(SLN), i cui layer sono i grafi [3-6] della  
            tabella \vref{tab:multilayer}. In queste reti i layer hanno tutti un numero simile 
            di nodi e archi;
    \item Different Layers Networks(DLN), i cui layer sono i grafi [1-3] della tabella 
        \vref{tab:multilayer}. In queste reti i layer hanno numero di nodi e archi molto differenti.
\end{itemize}

\paragraph{Generazione delle \interc}
Per prima cosa, sono state aggiunte \interc\ tra le controparti di ogni nodo in tutti 
i layer in cui esso è presente.

Quindi, sono state generate casualmente altre \interc\ impostando tre parametri:
\begin{enumerate}
    \item il numero di \interc\ uscenti da un nodo in un determinato layer;
    \item la frequenza con cui un layer viene scelto come layer di destinazione di 
        una connessione;
    \item la frequenza con cui un nodo in un certo layer viene scelto come nodo 
        di destinazione. 
\end{enumerate}
Ognuno di questi parametri è stato generato secondo una distribuzione che segue 
la legge di Zipf \cite{zipf:humanb}, per cui la frequenza di un certo valore è inversamente
proporzionale al valore stesso. 

Tale distribuzione genera numeri tra 1 ed un valore massimo $m$ con diverse probabilità,
secondo il valore di una variabile $s \in [0, 1]$
che regola il grado di `asimmetria' dei valori generati:
per $s=0$ la probabilità di ogni valore è la stessa, mentre per  
$s=1$, si ha $\mathcal{P}(1) = 2 \cdot \mathcal{P}(2) = 3 \cdot \mathcal{P}(3) =\dots = m \cdot \mathcal{P}(m)$. 

Le variabili che regolano i tre parametri sono chiamate rispettivamete
$s_{degree}$, $s_{layer}$, $s_{node}$.
Per ognuna sono stati sperimentati i valori $0.3$ e $0.8$. 
Per quanto riguarda il numero di \interc\ di un nodo in un determinato layer
il valore massimo è stato impostato a $d \cdot \log_2{\sum_{i}{V_i}}$, con 
$d = 2$. Per gli altri paramentri, invece, tutti i layer devono essere selezionabili
così come tutti i nodi all'interno di un layer.

La probabilità che un nodo abbia $k$ \interc\ è $\mathcal{P}_{degree}(k)$.
Per la scelta del layer di destinazione si è operato come segue:
è stata generata una permutazione casuale dei layer; definita $pos(l)$
la posizione del layer $l$ nella permutazione, la probabilità che questo venga 
scelto come layer di destinazione è $\mathcal{P}_{layer}(pos(l))$.
In modo analogo sono stati permutati i nodi all'interno di ogni layer per determinarne 
la probabilità di essere scelti come destinazione.

Le reti così generate sono denominate con 
SLN\large{$_{ s_{degree}\text{,}s_{layer}\text{,}s_{node}}$}
oppure
DLN\large{$_{ s_{degree}\text{,}s_{layer}\text{,}s_{node}}$}

L'algoritmo \vref{alg:generate} mostra come sono state generate le reti.

\SetStartEndCondition{ (}{)}{)}\SetAlgoBlockMarkers{}{\}}%
% \SetKwProg{Fn}{dsadsadas}{dsadad}
% \SetKwFunction{FRecurs}{generate\_multilayer}%
\SetKwFor{For}{for }{ \textbf{do}}{}{}%
\SetKwFor{If}{if }{ \textbf{then}}{}{}%
\SetKwFor{While}{while }{ \textbf{do}}{}{}%
% \SetKwIF{If}{ElseIf}{Else}{if}{\{}{elif}{else\{}{}%
% \SetKwFor{While}{while}{\{}{}%
% \SetKwRepeat{Repeat}{repeat\{}{until}%
% \AlgoDisplayBlockMarkers
% \SetAlgoNoLine%
\begin{algorithm}   
    \caption{Generazione multilayer networks}
    \label{alg:generate}
% \Fn{\FRecurs{asd}}
    % {
    \SetKwData{Left}{left}
    \SetKwData{This}{this}
    \SetKwData{Up}{up}
    \SetKwFunction{Union}{Union}
    \SetKwFunction{FindCompress}{FindCompress}
    
    \SetKwInOut{Input}{Input}
    \SetKwInOut{Output}{Output}
    \SetStartEndCondition{ }{}{}%
    \DontPrintSemicolon
    \Input{
        \textsc{Graph} $layers$\abracks \algcmnt{grafi usati come layers} \\ 
        \tint\ $perm\_l$\abracks \algcmnt{posizione di ogni layer nella permutazione}\\
        \tint\ $perm\_n$\abracks \abracks \algcmnt{posizione di un nodo nella permutazione di ogni layer}\\
        \tint\ $d $\\
        \tfloat\ $s\_degree$ \\
        \tfloat\ $s\_layer$ \\
        \tfloat\ $s\_node$ \\
    }
    
    \tint\ $total\_nodes = 0$\;
    \For{$i = 0$ \KwTo $layers.size()$} {
            $total\_nodes = total\_nodes + layers[i].nodes().size()$\;
    }
    \tint\ $max\_interconnections = d \cdot \lfloor\log_{2}{(total\_nodes)}\rfloor$\;
    \BlankLine
    
    \tcp{I due parametri della classe \textsc{ZipfGenerator} regolano rispettivamente \\
    il massimo della distribuzione e il grado di asimmetria}
    \tzgen\ $degree\_generator(max\_interconnections, s\_degree)$\;
    \tzgen\ $layer\_generator(layers.size(), s\_layer)$\;
    \tzgen\ $node\_generators[0\dots layers.size() - 1]$\;
    \For{$l = 0$ \KwTo $layers.size() - 1$}{
        $node\_generators[l] = $ \tzgen $(layers[l].size(), s\_degree)$\;
    }
    \BlankLine

    \textsc{MultilayerNetwork} $m$\;


    \For {$l = 0$ \KwTo $layers.size() - 1$}{
        \ForEach {$n \in layers[l]$}{
            \tcp*{intra-connessioni}
            \For {$v \in layers[l].adj(n)$}{
                $m.add\_edge(n, l, v, l)$\;
            }
            \tcp*{inter-connessioni tra un nodo e le sue controparti negli altri layer}
            \For {$j = 0$ \KwTo $layers.size() - 1$}{ 
                \If {$ j \neq l$ \textup{\textbf{and}} $n \in layers[j].nodes()$} {
                    $m.add\_edge(n, l, n, j)$\;
                }
            }
            \tcp*{Generazione inter-connessioni casuali}
            \tint\ $degree = degree\_generator.next()$\;
            \For {$i = 1$ \KwTo $degree$}{
                \tbool\ $added = \False$\;
                \While{\textup{\textbf{not}} $added$}{
                    \tint\ $l\_dest = l\_index[layer\_generator.next()]$\;
                    \While {$l\_dest == l$} {
                        $l\_dest = l\_index[layer\_generator.next()]$\;
                    }
                    \tint\ $n\_dest= n\_index[l\_dest][node\_generators[l\_dest].next()]$\;
                    \If{\textup{\textbf{not}} $m.has\_edge(n, l, n\_dest, l\_dest)$}{
                        $added = \True$\;
                        $m.add\_edge(n, l, n\_dest, l\_dest)$\;
                    }
                }
            }
        }
    }
\end{algorithm}
  
%   $\text{SLN}_d(s_{degree}$, $s_{layer}$, $s_{node})$
% oppure $\text{DLN}_d(s_{degree}$, $s_{layer}$, $s_{node})$.
% dataset multilayer
\begin{table}
    \caption{Dataset per la generazione di reti multilayer}
    \label{tab:multilayer}
    \centering
    \begin{tabular}{rlrr}
        \toprule
            Numero & Nome & Nodi & Archi \\
            % Nodi iniziali & Nodi estratti in  & Nodi estratti & Archi iniziali & Archi estratti in  & Archi estratti & Layer iniziali & Layer estratti \\
        \midrule
        1. & wiki-Vote & \num{7115} & \num{103689} \\
        2. & cit-HepTh & \num{27770} & \num{352807} \\
        3. & p2p-Gnutella04 & \num{10876} &  \num{39994} \\ 
        4. & p2p-Gnutella05  & \num{8846} & \num{31839} \\ 
        5. & p2p-Gnutella06  & \num{8717} & \num{31525} \\ 
        6. & p2p-Gnutella08 & \num{6301} & \num{20777} \\
        \bottomrule
    \end{tabular}
%         
\end{table}