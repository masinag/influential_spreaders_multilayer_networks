%%%%%%%%%%%%%%
% definizioni
\usepackage{amsmath}
\usepackage{amsthm}
\usepackage{amssymb}
\usepackage{mathrsfs}

%%%%%%%%%%%
% vref
\usepackage[italian]{varioref}
\usepackage{hyperref}
\hypersetup{
    colorlinks,
    citecolor=black,
    filecolor=black,
    linkcolor=black,
    urlcolor=black
}

\usepackage{url}


%%%%%%%%%%%%%%%
% algoritmi
\usepackage[ruled,linesnumbered,vlined,scleft,italiano]{algorithm2e}
\usepackage{eqparbox}
\newcommand{\algcmnt}[1]{%
  \texttt{\hfill// \eqparbox{COMMENT}{#1}}%
}
\newcommand{\alglcmnt}[1]{%
  \texttt{
    \hfill// \begin{minipage}[t]{
      \eqboxwidth{COMMENT}
    }#1\strut\end{minipage}
  }
}
\SetKwRepeat{Do}{do}{while}
% \newcommand{\forcond}{$i=0$ \KwTo $n$}
% \SetKwProg{Fn}{Function}{}{end}
% \SetKwFunction{FRecurs}{FnRecursive}%
\newcommand{\tint}{\texttt{int}}
\newcommand{\tfloat}{\texttt{float}}
\newcommand{\tbool}{\texttt{bool}}
\newcommand{\tzgen}{\textsc{ZipfGenerator}}
\newcommand{\False}{\texttt{False}}
\newcommand{\True}{\texttt{True}}

% \usepackage{algorithm}
% \usepackage{algorithmicx}
% \usepackage[noend]{algpseudocode}
% \def\NoNumber#1{{\def\alglinenumber##1{}\State #1}\addtocounter{ALG@line}{-1}}
% \floatname{algorithm}{}
\newcommand{\abracks}{[\,]}

%%%%%%%%%%%%%%%%
% tabelle
\usepackage{array}
\usepackage{multirow}
\usepackage{booktabs}
% \usepackage{caption}
\usepackage{siunitx} % formattazione numeri 
\usepackage[dvipsnames]{xcolor}
\newcommand{\1}[1]{\colorbox{Cerulean!50}{#1}}
\newcommand{\2}[1]{\colorbox{Cerulean!35}{#1}}
\newcommand{\3}[1]{\colorbox{Cerulean!20}{#1}}

%%%%%%%%%%%%%%
% immagini
\usepackage{graphicx}
\usepackage{graphics}
\usepackage{chngcntr}
\counterwithout{figure}{chapter}


%%%%%%%%%%%%%%%%
% definizioni

\theoremstyle{definition}
\newtheorem{definizione}{Definizione}
\theoremstyle{plain}
\newtheorem{teorema}{Teorema}

% \newcommand{\asd}{\href{asd}}

% definizioni
\newcommand{\infsp}{\textit{influential spreaders}}
\newcommand{\layers}{\textit{layers}}
\newcommand{\muln}{\textit{multilayer network}}
\newcommand{\mulx}{\textit{multiplex network}}
\newcommand{\gragg}{\textit{grafo aggregato}}
\newcommand{\interc}{\textit{inter-connessioni}}
\newcommand{\spproc}{\textit{spreading process}}
\newcommand{\epprob}{\textit{epidemic probability}}
\newcommand{\crepp}{\textit{critical epidemic point}}