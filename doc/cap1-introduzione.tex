\chapter{Introduzione}

Le reti permettono di modellare una grande quantità di sistemi in cui diverse entità,
anche dette nodi, sono legate tra loro mediante delle relazioni.
Esempi di questo tipo sono le reti di interazione sociale, in cui le persone sono legate 
da relazioni di amicizia, di collaborazione o di contatto sui social networks, 
le reti dei trasporti, come quello ferroviario o quello aereo
che collegano diverse città, e le reti biologiche, come quella alimentare o
quella delle interazioni tra molecole all'interno di un organismo.

Utilizzando questi modelli è particolarmente utile lo studio dei processi di diffusione
all'interno delle reti per comprendere le dinamiche di diversi scenari.
Tra questi, la diffusione di un virus in una società, di una notizia in una rete di contatti, 
una fake news in un social network e l'accumulo di un ritardo nelle stazioni di una rete 
dei trasporti.

In questi processi, inizialmente un nodo o un ristretto gruppo di nodi sono infetti. In
ogni instante un nodo infetto può contagiare con una certa probabilità i nodi con cui 
è in relazione.
Di particolare interesse è l'individuazione dei cosiddetti \infsp, ovvero dei nodi
che, se infettati, permettono di `contagiare' una grossa parte della rete.
Tali nodi, infatti, possono essere sfruttati per garantire una efficace diffusione 
di un'informazione nella rete o, al contrario, per evitarne la propagazione.
Ad esempio, per diffondere una notizia importante in un social network sarà opportuno che 
venga pubblicata da un account molto seguito, mentre per fermare un'epidemia 
può essere utile isolare i luoghi o le persone che hanno un ruolo più centrale
nella società.


\section{\textit{Influential spreaders} in un grafo}

Generalmente, una rete viene modellata tramite un grafo, ossia una struttura in cui i 
nodi sono connessi tramite archi se tra loro esiste una relazione.

Il modo più affidabile per l'individuazione degli \infsp\ in un grafo consiste nel 
simulare il processo di diffusione partendo da ogni nodo, classificando come nodi più 
influenti quelli da cui si riesce ad infettare la porzione di rete maggiore.
Tuttavia, poichè la trasmissione dell'infezione tra un nodo e un suo vicino è probabilistica,
occorrerebbe effettuare un gran numero di simulazioni prima di avere una stima accurata 
dell'influenza di un nodo sulla rete e ciò rende questo metodo computazionalmente 
molto costoso.

Un problema molto studiato è quello dell'individuazione degli \infsp\ tramite l'analisi della 
topologia delle rete\cite{kitsak:infsp}\cite{basaras:infsp}\cite{pei:infsp}.
È stato osservato che alcuni algoritmi di centralità forniscono una buona indicazione di 
quali sono i nodi più influenti nella diffusione. 
Tra questi, \textit{PageRank}, \textit{Betweenness Centrality} e \textit{k-core} sono algoritmi che richiedono una conoscenza
globale della topologia della rete, mentre per altri come Degree Centrality o $\mu$-PCI ogni nodo 
ha bisogno solo di una conoscenza locale della rete.

\section{\textit{Influential spreaders} in una multilayer network}

In molti dei problemi reali i nodi sono collegati tramite diversi tipi di relazioni, che non 
possono essere rappresentate in un grafo `classico'. 
Esempi di questo tipo sono la rete dei profili online collegati dalle loro interazioni
in diversi social networks (follow, amicizie, condivisioni, ecc.) e la rete delle stazioni di 
una città connesse da tratte di diversi mezzi di trasporto (rete ferroviaria, rete degli autobus, 
piste ciclabili, ecc.).

Strutture che permettono di rappresentare queste realtà sono le \mulns, ossia reti in cui sono 
presenti differenti tipi di relazione tra i nodi. 

È stata studiata l'individuazione degli \infsp\ in questo tipo di reti, utilizzando 
algoritmi di centralità definiti per esse.

