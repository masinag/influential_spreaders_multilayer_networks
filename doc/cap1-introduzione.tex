\chapter{Introduzione}

Lo studio di un processo di diffusione all'interno di una rete è molto utile per comprendere le dinamiche
di diversi scenari. Tra questi, la diffusione di una notizia in una rete di contatti, di un virus in una 
società, una fake news in un social network e l'accumulo di un ritardo nelle stazioni di una rete di trasporti.

Di particolare interesse è l'individuazione dei cosiddetti \infsp, ovvero dei nodi
che, se infettati, permettono di `contagiare' una grossa parte della rete.

\section{\textit{Influential spreaders} in un grafo}

Poichè i processi di simulazione della diffusione sono computazionalmente costosi, un problema
di grande importanza è quello di individuare gli \infsp\ analizzando la topologia della rete.

Il problema è stato molto studiato nei grafi `classici', dove i nodi e gli archi che li collegano 
appartengono alla stessa rete %
\cite{kitsak:infsp}\cite{basaras:infsp}\cite{pei:infsp}.
È stato osservato che alcuni algoritmi di centralità forniscono una buona indicazione di 
quali sono i nodi più influenti nella diffusione. 
Tra questi, \textit{PageRank}, \textit{Betweenness Centrality} e \textit{k-core} sono algoritmi che richiedono una conoscenza
globale della topologia della rete, mentre per altri come Degree Centrality o $\mu$-PCI ogni nodo 
ha bisogno solo di una conoscenza locale della rete.

\section{\textit{Influential spreaders} in una multilayer network}

In molti dei problemi reali i nodi sono collegati tramite diversi tipi di interazioni, che non 
possono essere rappresentate in un grafo `classico'. 
Esempi di questo tipo sono la rete dei profili online collegati dalle loro interazioni
in diversi social networks (follow, amicizie, condivisioni, ecc.) e la rete delle stazioni di 
una città connesse da tratte di diversi mezzi di trasporto (rete ferroviaria, rete degli autobus, 
piste ciclabili, ecc.).

Strutture che permettono di rappresentare queste realtà sono le \muln, ossia reti in cui sono 
presenti differenti tipi di relazione tra i nodi. 

È stata studiata l'individuazione degli \infsp\ in questo tipo di reti, utilizzando 
algoritmi di centralità definiti su di esse.

