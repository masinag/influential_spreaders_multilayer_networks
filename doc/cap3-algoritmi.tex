\chapter{Algoritmi utilizzati}
Dopo una fase di ricerca, sono stati selezionati ed implementati alcuni algoritmi di centralità solitamente usati per 
l'individuazione degli \infsp. Tali algoritmi assegnano ad ogni nodo un valore proporzionale alla sua centralità e 
quindi all'influenza esercitata da tale nodo sulla rete.

Questi possono essere divisi in tre categorie:
\begin{itemize}
\item algoritmi applicati sul grafo aggregato;
\item algoritmi applicati separatamente su ogni layer;
\item algoritmi applicati sull'intera struttura \textit{multilayer}.
\end{itemize}

\section{Algoritmi sul grafo aggregato}
Diverse misure di centralità sono utilizzate per l'individuazione degli \infsp\ in grafi 
`classici' ad un solo layer, detti anche \textit{monoplex}~\cite{kitsak:infsp}\cite{basaras:infsp}\cite{pei:infsp}. 
Essendo definite per grafi ad un solo layer, sono state applicate al \gragg\
derivato da ogni \muln. 
Le misure utilizzate sono state \PageRank~\cite{page:pagerank}, 
\kcore~\cite{batagelj:kcore} e 
\degree.

% PAGE RANK

\subsection{\PageRank}
\label{alg:pagerank}
\begin{definizione}[\PageRank]
    \label{def:pagerank}
    Dato un grafo $G=(V, E)$, il valore di \PageRank\ di un nodo $v$ è definito come
    \begin{equation}
        PR(v) = \alpha\sum_{w : v \in G.adj(w)}
        \frac{PR(w)}{G.outdegree(w)} + 
        (1-\alpha)\frac{1}{|V|}
    \end{equation}
    con $0 \le \alpha \le 1$.
\end{definizione}

Questa misura di centralità fu originariamente definita per misurare l'importanza di una pagina web. 
Secondo questo criterio, una pagina è tanto più importante quante più pagine importanti hanno un link 
verso di essa. 
\PageRank\ può essere visto come un modello di comportamente di un utente che, trovandosi inizialmente
in una pagina casuale, continua a visitare i link che trova, ma occasionalmente riparte da un'altra 
pagina casuale.
Il parametro $\alpha$ è detto $damping factor$ e regola la probabilità che si riparta da una pagina 
casuale ed è stato impostato a \num{0.85}.

Analogamente può essere utilizzata per misurare più genericamente l'influenza di un nodo 
in un grafo.

Questa definizione ricorsiva è stata così calcolata in modo iterativo: 

% \begin{equation}
%     \begin{split}
% PR_0(v)&= PR(v) \text{ in } X_1 \\
% addPR_l(v)&= \alpha \sum_{w : v \in X_l.adj(w)} 
%         \frac{addPR_l(w)}{X_l.outdegree(w)} + 
%         (1-\alpha)\frac{addPR_{l-1}(v)}{|V| \langle addPR_{l-1} \rangle}
%     \end{split}
% \end{equation}
\begin{equation}
    \begin{split}
PR_0(v)&= \frac{1}{n}\\
PR_i(v)&= \alpha \sum_{w : v \in G.adj(w)} 
\frac{PR_{i-1}(w)}{G.outdegree(w)} + 
(1-\alpha)\frac{1}{|V|}
    \end{split}
\end{equation}
finchè $||PR_{i} - PR_{i-1}||_1 < \epsilon$. Secondo \cite{page:pagerank} il numero di iterazioni 
è proporzionale a $\log{|V|}$. Dunque, la complessità dell'algoritmo è $O((|V| + |E|)\log{|V|})$.

% K-CORE

\subsection{\kcore}
\begin{definizione}[\kcore]
    \label{def:kcore}
    Dato un grafo $G=(V, E)$, un \kcore\ di $G$ è un sottografo $G'=(V',E')$
    tale che 
    \begin{equation}
        G'.indegree(v) \geq k \quad \forall v \in V'
    \end{equation}
\end{definizione}

Un nodo appartenente ad un \textit{k-core} con $k$ elevato è considerato un nodo centrale 
nella rete, e quindi un potenziale \textit{influential spreader}.
Per il calcolo di questa misura è stato utilizzato l'algoritmo definito in~\cite{batagelj:kcore}
con complessità $O(|V| + |E|)$.

\subsection{\degree}
Questa misura di centralità assegna ad ogni nodo un valore pari al suo \textit{outdegree}, 
per cui un nodo con tanti vicini si assume possa avere una certa rilevanza in un 
processo di diffusione.
Può essere calcolata con complessità $O(|V|)$.

\section{Algoritmi applicati su ogni layer}
Alcune delle misure sono state estese alle reti multilayer calcolando la centralità
dei nodi in ogni layer separatamente e poi sommando i punteggi ottenuti in ogni layer.
In questa categoria rientrano \addPageRank~\cite{halu:addpagerank} 
e \sumCore~\cite{basaras:infspmul}

\subsection{\addPageRank}
\begin{definizione}[\addPageRank]
    Data una \muln\ $\mathcal{M}=(\mathcal{G}, \mathcal{E})$ 
    ed un ordinamento dei layer $(X_1, \dots, X_{L})$, l'\addPageRank\
    di un nodo $v \in V$ è il valore 
    $addPR(v) = addPR_L(v)$, dove:

    \begin{equation}
        \begin{split}
addPR_1(v)&= PR(v) \text{ in } X_1 \\
addPR_l(v)&= \alpha \sum_{w : v \in X_l.adj(w)} 
            \frac{addPR_l(w)}{X_l.outdegree(w)} + 
            (1-\alpha)\frac{addPR_{l-1}(v)}{|V| \langle addPR_{l-1} \rangle}
        \end{split}
    \end{equation}

\end{definizione}

Questa estensione dell'algoritmo di PageRank alle \muln\ richiede un ordinamento
dei layer. Come fatto in \cite{basaras:infspmul}, i layer sono stati ordinati per valore
dell'autovalore di modulo massimo della rispettiva matrice di adiacenza. Infatti, 
autovalore più grande significa maggiore capacità di diffusione~\cite{wang:eigenv}.

Come algoritmo, è stato utilizzato un approccio analogo a quello della versione classica di 
PageRank descritto in~\vref{alg:pagerank}.
Poichè è necessario calcolare il PageRank su ogni layer, il costo dell'algoritmo è 
$O(\sum_{i}((|V_i| + |E_i|)log(|V_i|)))$, assumendo che i layer siano già ordinati.

\subsection{\sumCore}
Questa misura è stata ottenuta calcolando il valore di \kcore di ogni nodo in 
ogni layer, quindi sommando i valori ottenuti da un nodo in tutti i layer.
Pertanto la complessità del calcolo è pari a $O(\sum_{i}(|V_i| + |E_i|)$.

\section{Algoritmi applicati sull'intera struttura}
Diverse misure definite originariamente per grafi \textit{monoplex} sono state estese a 
reti \textit{multilayer}. Queste, al contrario di quelle presentate nelle sezioni precedenti,
tengono conto della struttura a più livelli della rete ed in particolare delle \interc.
Quelle implementate sono 
\verPageRank~\cite{dedomenico:versatile},
\verBetweennessCentrality~\cite{dedomenico:versatile} \cite{dedomenico:verbetw},
\multiCore~\cite{azimi:multikcore} e le diverse varianti di \PCI: 
\mlPCI~\cite{basaras:infspmul},
\laPCI~\cite{basaras:infspmul},
\alPCI~\cite{basaras:infspmul},
\lsPCI~\cite{basaras:infspmul}.

\subsection{\verPageRank}
\begin{definizione}[\verPageRank]
    Data una \muln\ $\mathcal{M}=(\mathcal{G}, \mathcal{E})$,
    si definisce \verPageRank\ di un nodo $v \in V_l$ in un layer $l$
    il valore:

    \begin{equation}
        PR(v, l) = 
        \alpha \biggl( 
            \sum_{w : v \in G_l.adj(w)} \frac{PR(w, l)}{\mathcal{M}.outdegree(w, l)} +
            \sum_{j \neq l}^{}\sum_{w : (w, v) \in E_{jl}} \frac{PR(w, j)}{\mathcal{M}.outdegree(w, j)} 
        \biggr) + (1-\alpha)\frac{1}{N}
    \end{equation}
    dove $\mathcal{M}.outdegree(w, l)$ è la somma del numero di archi uscenti da $w$ nel layer $l$
    e del numero di \interc\ uscenti da $w$ nel layer $l$.

\end{definizione}

La definizione originale in \cite{dedomenico:versatile} utilizzava un tensore 4-dimensionale per 
rappresentare la rete. Qui è stata generalizzata, anche prevedendo che un nodo possa non comparire 
in tutti i layer.

Questa definizione è stata calcolata in modo analogo alla definizione \vref{alg:pagerank},
dunque la complessità è $O((N + E)\log(N))$.

\subsection{\verBetweennessCentrality}

\begin{definizione}[\verBetweennessCentrality]
    Data una \muln\ $\mathcal{M}=(\mathcal{G}, \mathcal{E})$, la 
    \verBetweennessCentrality\ di un nodo $v$ è il valore 
    \begin{equation}
        BC(v) = \sum_{s, t \in V} \frac{\sigma_{st}(v)}{\sigma_{st}}
    \end{equation}
    dove $\sigma_{st}$ è il numero di percorsi minimi tra il nodo $s$ ed il nodo $t$
    indipendentemente dal layer, e $\sigma_{st}(v)$ è il numero di questi che passa 
    per il nodo $v$ in qualche layer.
\end{definizione}

Questa misura è stata calcolata con l'algoritmo definito in \cite{dedomenico:verbetw}, 
con complessità $O\left(|V|\left(N+E\right)\right)$.

\subsection{\multiCore}

\begin{definizione}[\multiCore]
    Data una \muln\ $\mathcal{M}=(\mathcal{G}, \mathcal{E})$, il \multiCore\ è
    il più grande sottografo per cui ogni nodo è raggiunto in ogni layer da almeno $k$ archi o
    \interc.
\end{definizione}

Questa misura è stata calcolata estendendo l'algoritmo definito in \cite{batagelj:kcore} per 
adattarlo a questa definizione, ottenendo una complessità di $O(N + E)$.

\subsection{\PCI}
\begin{definizione}
    Data una \muln\ $\mathcal{M}=(\mathcal{G}, \mathcal{E})$, 
    l'indice $mlPCI_n$ di un nodo $v$ in un layer $l$ 
    è il massimo numero $k$ tale che esistono almeno $k$ vicini di $v$ nel layer $l$ in $\mathcal{M}$
    con numero di vicini in almeno $n$ layer maggiore o uguale a $k$.
\end{definizione}

Da questa definizione si ricavano le seguenti:

\begin{definizione}[\mlPCI]
    L'indice $mlPCI$ di un nodo $v$ in un layer $l$ è definito come 
    \begin{equation}
        mlPCI(v, l) = \sum_{n=1}^L mlPCI_n(v, l)
    \end{equation}
\end{definizione}

\begin{definizione}[\alPCI]
    L'indice $alPCI$ di un nodo $v$ in un layer $l$ è definito come 
    \begin{equation}
        alPCI(v, l) = mlPCI_L(v, l)
    \end{equation}
\end{definizione}


\begin{definizione}[\lsPCI]
    L'indice $lsPCI$ di un nodo $v$ in un layer $l$ è il massimo numero $k$ 
    tale che esistono almeno $k$ vicini di $v$ nel layer $l$ in $\mathcal{M}$
    con numero di vicini in almeno $k$ layer maggiore o uguale a $k$.
\end{definizione}

\begin{definizione}[\laPCI]
    L'indice $laPCI$ di un nodo $v$ in un layer $l$ è 
    è il massimo numero $k$ tale che esistono almeno $k$ vicini di $v$ nel layer 
    $l$ in $\mathcal{M}$
    con numero di vicini maggiore o uguale a $k$.
\end{definizione}



Il calcolo di queste definizioni, è stato fatto utilizzando un algoritmo di complessità 
$O(N L t \log{t})$, dove $t = \max{\{\mathcal{M}.outdegree(v)\,|\,v \in V\}}$ 
