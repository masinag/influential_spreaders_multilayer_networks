\setlength{\parskip}{1em}

\chapter*{Sommario} % senza numerazione
\label{sommario}

\addcontentsline{toc}{chapter}{Sommario} % da aggiungere comunque all'indice

Diversi sistemi reali possono essere modellati tramite una rete in cui
le entità, sono legate se tra loro c'è una relazione. 
In particolare in molti casi le relazioni tra queste entità possono essere 
di varia natura. Questo tipo di sistema può essere rappresentato usando una 
\muln.

Lo scopo di questa tesi è l'individuazione degli \infsp\ in una \muln, 
ovvero di quei nodi nella rete che durante un processo di diffusione 
permettono di infettare la porzione di rete maggiore. 
Questi nodi sono particolarmente utili in quanto possono essere sfruttati per 
massimizzare la diffusione di un'informazione all'interno della rete o, al contrario, 
isolati se si vuole che l'informazione resti il più possibile confinata.

% \\[1em]
Tali nodi sono facilmente identificabili simulando i processi di diffusione, 
che però risultano computazionalmente molto costosi e quindi 
non applicabili a reti molto grandi.
Pertanto l'obiettivo è stato quello di individuarli analizzando la topologia 
della rete.
% \\[1em]
Una fase di ricerca iniziale ha permesso di raccogliere, studiare 
e successivamente implementare diversi 
algoritmi di centralità già utilizzati per risolvere questo problema 
sia in grafi che in \mulns.

Questi algoritmi sono stati applicati a diversi tipi di reti \textit{multilayer}:
reti \textit{multiplex} estratte a partire da dataset biologici e 
sociali reali e reti \textit{multilayer} generate casualmente 
a partire da grafi derivati da applicazioni peer-to-peer.

I valori di centralità dei nodi restituiti da ogni algoritmo sono stati 
confrontati con quelli reali, ottenuti simulando dei processi di diffusione
nelle varie reti. In questo modo è stato possibile assegnare ad ogni algoritmo
un punteggio ottenuto su ciascuna rete e individuare così quali algoritmi
possono essere utilizzati per trovare gli \infsp\ in una \muln.
Due due gli algoritmi testati hanno ottenuto buone performance su tutte le 
reti analizzate, mentre le prestazioni degli altri sono state scarse o altalenanti.

% Il sommario dell’elaborato consiste al massimo di 3 pagine e deve contenere le seguenti informazioni:
% \begin{itemize}
%   \item contesto e motivazioni 
%   \item breve riassunto del problema affrontato
%   \item tecniche utilizzate e/o sviluppate
%   \item risultati raggiunti, sottolineando il contributo personale del laureando/a
% \end{itemize}



