\chapter{Simulazioni}

Per valutare se un algoritmo possa essere utilizzato per riconoscere gli \infsp, 
sono stati confrontati i valori di centralità che esso assegna ai vari nodi con la frazione di 
rete infettata da essi simulando un processo di diffusione, detto \spproc.

Lo \spproc\ utilizzato segue il modell \textit{SIR}, in cui un nodo può essere in 3 stati:
\begin{itemize}
    \item \textit{Susceptible} (\textit{S}) - ossia vulnerabile ad essere `infettato';
    \item \textit{Infectious} (\textit{I}) - ossia `infetto'. Un nodo in questo stato può contagiare
        i suoi vicini con una certa probabilità;
    \item \textit{Recovered} (\textit{R}) - ossia guarito. Una volta in questo stato, un nodo non può più essere
        infettato.
\end{itemize}

\section{Probablità di contagio}
Un valore da impostare in uno \spproc\ è la probabilità $\lambda$ che un nodo nello stato \textit{I}
ha di contagiare i propri vicini nello stato \textit{S}, detta \epprob.
La scelta del valore di $\lambda$ è molto importante per riconoscere gli \infsp, 
in quanto con un valore troppo alto si osserverebbe la diffusione dell'infezione in tutta la 
rete indipendentemente dal nodo inizialmente infetto, mentre con un valore troppo basso l'epidemia 
non riuscirebbe ad espandersi nemmeno dai nodi più influenti. 

In un grafo \textit{monoplex} $G$ il valore di \crepp\
\begin{equation}
    \lambda_c = \frac{\langle k  \rangle}{\langle k^2 \rangle}
\end{equation}
dove $k$ è l'\textit{outdegree} di un nodo, rappresenta un'approssimazione della soglia per 
cui, indipendentemente dal nodo da cui parte l'epidemia, se $\lambda > \lambda_c$ essa riesce 
a diffondersi in gran parte della rete, mentre se $\lambda < \lambda_c$ essa rimane confinata 
in una piccola parte della rete\cite{saumell:epidemicsp}.

Seguendo quanto fatto in \cite{basaras:infspmul}, la \epprob\ $\lambda_{ii}$ da un nodo nel 
layer $i$ ad uno dello stesso layer è stata impostata al \crepp\ del layer $G_i$, 
mentre $\lambda_{ij}$ tra nodi in layer diversi è stata impostata al \crepp\ del grafo 
aggregato. 

Questi valori rispecchiano l'intuizione secondo cui il contagio si trasmette 
più facilmente tra nodi sullo stesso layer. Questo succede per esempio nel caso dei social 
network, in cui diverse piattaforme tendono a mostrare in maniera minore contenuti con
collegamenti a piattaforme concorrenti, e nel caso dei ritardi accumulati da 
una stazione in una rete di trasporti, che possono trasmettersi facilmente alle stazioni vicine 
nella stessa rete (es. rete ferroviaria), ma anche, indirettamente e presumibilmente in maniera
minore, ad altre reti di cui la stazione fa parte.